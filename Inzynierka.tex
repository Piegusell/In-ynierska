\documentclass[12pt,a4paper]{article}
\usepackage[pagestyles]{titlesec}

\usepackage{adjustbox}
\usepackage{amsbsy}
\usepackage{amsmath}
\usepackage{amsfonts}
\usepackage[toc,page]{appendix}
\usepackage{array}
\usepackage[polish]{babel}
\usepackage{booktabs}
\usepackage{bbm}
\usepackage{bbold}
\usepackage[makeroom]{cancel}
\usepackage{cellspace}
\usepackage{collcell}
\usepackage{colortbl}
\usepackage{enumitem}
\usepackage[intlimits]{esint}
\usepackage{fancyhdr}
\usepackage{fancyvrb}
\usepackage{float}
\usepackage[a4paper,portrait, margin=1in]{geometry}
\usepackage{graphicx}
\usepackage{hyperref}
\usepackage[utf8]{inputenc}
\usepackage{lastpage}
\usepackage{listings}
\usepackage{longtable}
\usepackage{makecell}
\usepackage{multicol}
\usepackage{multirow}
\usepackage{newverbs}
\usepackage{pdfpages}
\usepackage{setspace}
\usepackage{subcaption}
\usepackage{tabularx}
\usepackage{tikz}
\usetikzlibrary{matrix,calc}
\usepackage{todonotes}
\usepackage{ulem}
\usepackage{wasysym}
\usepackage{wrapfig}
\usepackage{xcolor}
\usepackage{xfrac}
\usepackage{yfonts}


\setlength{\cellspacetoplimit}{6pt}
\setlength{\cellspacebottomlimit}{6pt}
\setlength{\parindent}{0.95cm}


\newcolumntype{M}[1]{>{\centering\arraybackslash}m{#1}}
\newcolumntype{C}[1]{>{\centering\arraybackslash}S{m{#1}}}
\newcolumntype{N}[1]{>{\raggedright\arraybackslash}S{m{#1}}}
\newcolumntype{L}[1]{>{\raggedright\arraybackslash}m{#1}}

\DefineVerbatimEnvironment{Code}{BVerbatim}{numbers=left}
\definecolor{asparagus}{rgb}{0.34, 0.58, 0.14}
\definecolor{asparagus2}{rgb}{0.61, 0.76, 0.49}
\definecolor{ziel}{rgb}{0.29, 0.59, 0.09}

\lstnewenvironment{todoverbatim}[1][]
{\lstset{
		basicstyle=\ttfamily,
		columns=flexible,
		escapeinside = {/*}{*/}
}}
{}

\renewcommand\fbox{\fcolorbox{lightgray}{white}}


\newcommand{\up}{\color{white}$\frac{|^|}{o}$ \color{black}}
\newcommand{\upm}{\color{red}\frac{|}{|}\color{black}}
\newcommand{\uu}[1]{\overset{\color{white}o\color{black}}{#1}}
\newcommand*{\Resize}[2]{\resizebox{#1}{!}{$#2$}}
% ------------------------------
% Autor i tytuł
% ------------------------------
\author{Wiktor Muehsam}
\title{Praca dyplomowa}


\begin{document}
\input{content/Strona_tytulowa.tex}
\newpage
\tableofcontents

\newpage


\section{Wstęp}\label{wstep}
Reaktor MARIA został uruchomiony w grudniu 1974 roku jako część krajowego programu rozwoju energetyki jądrowej. Był to pierwszy reaktor badawczy w Polsce zdolny do pracy w trybie ciągłym, przeznaczony do prowadzenia eksperymentów naukowych, produkcji izotopów promieniotwórczych oraz testowania materiałów jądrowych. 
	
W ciągu kolejnych dekad reaktor przeszedł szereg modernizacji, które zwiększyły jego bezpieczeństwo, wydajność oraz zakres możliwych zastosowań. Obecnie jest jednym z 20 reaktorów badawczych, które produkują izotopy promieniotwórcze w celach medycznych. Aby zachować ciągłość jego pracy i zapewnić nieprzerwaną dostawę preparatów do placówek medycznych, potrzeba regularnych obliczeń i analiz bezpieczeństwa.

\subsection	{Cel i zakres}
Celem niniejszej pracy inżynierskiej jest analiza zachowania reaktora jądrowego w ujęciu kinetyki punktowej, w odpowiedzi na zmianę reaktywności, wywołaną ruchem pręta regulacyjnego.
W pracy zostaną przeprowadzone symulacje numeryczne z wykorzystaniem języka Python, polegające na rozwiązaniu układu równań kinetyki punktowej dla przypadku wprowadzenia dodatniej porcji reaktywności do rdzenia poprzez wysunięcie fragmentu pręta regulacyjnego.
Jak i zostanie wyznaczona charakterystyka efektywności pręta regulacyjnego w funkcji jego zagłębienia w rdzeniu.
Zostanie również wyznaczona wyprowadzone zostanie tzw. „równanie odwrotnych godzin” jako szczególny przypadek rozwiązania równań kinetyki punktowej, przy założeniu wprowadzenia do rdzenia dodatniej porcji reaktywności. Równanie to zostanie wykorzystane w metodzie przeprowadzania pomiaru efektywności pręta regulacyjnego.
Na podstawie tego pomiaru zostanie wyznaczona eksperymentalna charakterystyka efektywnościowa pręta w funkcji jego zagłębienia w rdzeniu, którą następnie porówna się z wcześniej uzyskaną krzywą teoretyczną.
	



\section{Budowa reaktora i prętów pochłaniających - działanie i budowa}
„Rdzeń reaktora składa się z ciśnieniowych kanałów paliwowych, prętów regulacyjnych
i matrycy złożonej z bloków berylowych. Wokół rdzenia umieszczone są bloki grafitowe spełniające rolę reflektora. Całość umieszczona jest w obudowie zwanej koszem. Kosz posadowiony jest na specjalnej podstawie umieszczonej na dnie basenu reaktora. ” \cite{raport5}

W niniejszej pracy skupiono się głównie na prętach kontrolnych (\textbf{PK}), które stanowią podstawowy element układu regulacji poziomu mocy reaktora. Jak można zauważyć na rysunku \ref{rdzen}, rdzeń reaktora ma kształt stożka, przez co również pręty są wprowadzane pod kątem względem jego osi, co wpływ na regulację \textbf{PK} podczas eksploatacji.

\begin{figure}[H]
	\centering
	\includegraphics[scale = 0.7]{figures/geometria.png}
	\caption{Przekrój poprzeczny bloku reaktora z raportu bezpieczeństwa \cite{raport5}}
	\label{rdzen}
\end{figure}

\newpage
\subsection{Pręty kontrolne}
Pręt kontrolny wykonany jest z substancji pochłaniającej neutrony (węglika boru), zamkniętej w koszulce aluminiowej. Celem konstrukcji pręta jest kontrola reakcji rozszczepienia zachodzącej w rdzeniu reaktora poprzez wydajne pochłanianie neutronów. Zadania stawiane prętom kontrolnym są następujące:
\begin{itemize}
	\item utrzymanie głębokiej podkrytyczności, czyli stanu wyłączenia reaktora,
	\item podnoszenie mocy i praca przy mocy znamionowej,
	\item wyłączanie reaktora poprzez obniżanie mocy,
	\item awaryjne wyłączenie reaktora w przypadku zadziałania systemów zabezpieczeń.
\end{itemize}

Pręt kontrolny może pełnić trzy funkcje: kontrolną, bezpieczeństwa lub automatycznej regulacji.
Kanały prętów bezpieczeństwa (\textbf{PB}) i kompensacyjnych (\textbf{PK}), w tym także pręta automatycznej regulacji (\textbf{PAR}), umieszczone są w blokach berylowych. \cite{raport5}

Materiałem czynnym prętów jest węglik boru (B$_4$C), zwykle z podwyższoną zawartością izotopu $^{10}$B. Pochłanianie neutronów zachodzi głównie na reakcji wychwytu:
\[
^{10}\mathrm{B}+n \rightarrow {}^{7}\mathrm{Li}^{*}+\alpha+2{,}31~\mathrm{MeV}\quad (\approx 93{,}7\%),\ ,
\]
oraz, rzadziej,
\[
^{10}\mathrm{B}+n \rightarrow {}^{7}\mathrm{Li}+\alpha+2{,}79~\mathrm{MeV}\quad (\approx 6{,}3\%).
\]
  Po każdym akcie pochłonięcia neutronu przez $^{10}$B, jądro $^{10}$B jest zastępowane przez $^{7}$Li, a to oznacza, że z czasem $^{10}$B w danym pręcie pochłaniającym jest coraz mniej, czyniąc pręt mniej wydajnym.
  Dlatego tak ważna jest okresowa kontrola prętów poprzez pomiar, wykonany \textbf{metodą zrzutu} (rozdział\,\ref{metoda_zrzutu}) lub \textbf{metodą okresu} (rozdział\,\ref{metoda_okresu}).

\section{Równanie kinetyki punktowej}\label{punktowej}
\subsection{Klasyfikacja problemu czasowego}
Procesy zachodzące w reaktorze można podzielić na trzy grupy:
\begin{itemize} 
\item zjawiska długoterminowe trwające dni, lata (wypalenie paliwa, zatrucie bloków berylowych).
\item zjawiska o średniej stałej czasowej które trwają minuty, godziny (zatrucie ksenonem, czy samarem).
\item zjawiska bardzo szybkie, które zachodzą nawet w części sekundy (zrzucenie prętów kontrolnych, załączenie obiegu pierwotnego do chłodzenia reaktora, czy awarie z udziałem prętów pochłaniających, czy układu chłodzenia).
\end{itemize}
 Zakłada się, że reaktor pracuje jako punkt, stąd też nazwa kinetyka punktowa. Takie podejście to podstawa dla większości analiz systemów reaktorowych \cite{Lamarsh}. Równania kinetyki punktowej wyglądają następująco: 
\begin{equation}
	\frac{\mathrm{d}n(t)}{\mathrm{d}t}=\frac{\rho(t)-1)\beta}{\Lambda}n(t)+\sum_{i=1}^{j} \lambda_iC_i(t) 
	\label{nodt}
\end{equation}

\begin{equation}
	\frac{\mathrm{d}C_i(t)}{\mathrm{d}t}=\frac{\beta_i}{\Lambda}n(t)-\lambda_iC_i(t) 
	\label{ciodt}
\end{equation}

\noindent gdzie:

\noindent $n(t)$  jest zmieniającą się w czasie, $t$, gęstością neutronów,

\noindent $\rho_{\$}(t)$  jest reaktywnością wprowadzaną do reaktora, podaną w jednostkach~,\$


\noindent $\beta_i$  jest udziałem neutronów opóźnionych i fotoneutronów z $i$-tej grupy w całej puli neutronów, 

\noindent $\beta$   jest udziałem wszystkich neutronów opóźnionych i fotoneutronów w całej puli neutronów, $\beta~=~\sum\limits_{i=1}^j \beta_i$,

\noindent $\Lambda$  jest czasem życia jednej generacji neutronów,

\noindent $\lambda_i$   jest stałą rozpadu prekursorów neutronów opóźnionych i fotoneutronów z $i$-tej grupy,

\noindent $C_i(t)$ jest koncentracją prekursorów neutronów opóźnionych i fotoneutronów z $i$-tej grupy. 

Równania te możemy zapisać w postaci macierzowej, jako: 


\[
\frac{d}{dt}\;
\underbrace{\begin{bmatrix}
		n(t)\\[2pt]
		C_{1}(t)\\[2pt]
		C_{2}(t)\\[2pt]
		\vdots\\[2pt]
		C_{j}(t)
\end{bmatrix}}_{\vect f(t)}
\;=\;
\underbrace{\begin{bmatrix}
		\dfrac{(\rho_s(t)-1)\,\beta}{\Lambda} & \lambda_1 & \lambda_2 & \cdots & \lambda_j\\[3mm]
		\dfrac{\beta_1}{\Lambda} & -\lambda_1 & 0 & \cdots & 0\\
		\dfrac{\beta_2}{\Lambda} & 0 & -\lambda_2 & \ddots & \vdots\\
		\vdots & \vdots & \ddots & \ddots & 0\\
		\dfrac{\beta_j}{\Lambda} & 0 & \cdots & 0 & -\lambda_j
\end{bmatrix}}_{\mat M(t)}
\;
\underbrace{\begin{bmatrix}
		n(t)\\[2pt]
		C_{1}(t)\\[2pt]
		C_{2}(t)\\[2pt]
		\vdots\\[2pt]
		C_{j}(t)
\end{bmatrix}}_{\vect f(t)}\,.
\]


\section{Równanie odwrotnych godzin.}
W tym rozdziale, korzystając z modelu kinetyki punktowej przedstawionego wcześniej (rów.~\eqref{nodt}–\eqref{ciodt} oraz ujęcia macierzowego z macierzą $\mathbf{M}(t)$), wyprowadzimy zależność wiążącą reaktywność $\rho$ z okresem reaktora $T$.

Okres reaktora $T$ definiuje się jako czas, w którym strumień (moc) neutronowa rośnie $e$-krotnie: $n(t)=n_0\,e^{t/T}$. 
W praktyce często podaje się \emph{okres podwojenia} $T_2$, określony warunkiem $n(t)=2n_0$, przy czym
\[
T_2 = T\,\ln 2.
\]

\noindent\textbf{Równanie odwrotnych godzin} opisuje zależność reaktywności reaktora $\rho$ od jego okresu $T$. 
Nazwa pochodzi od „odwrotnej godziny” (ang.\ \emph{inverse hour}, \emph{inhour}) — jednostki reaktywności odpowiadającej sytuacji, w której okres reaktora $T$ wynosi jedną godzinę:
\begin{equation}
	\rho_{in}=\frac{\Lambda + \sum\limits_{i=1}^j \sfrac{\beta_i}{\lambda_i}}{3600}\,\rho
	\label{IN}
\end{equation}

\noindent gdzie:
\begin{itemize}
	\item $\rho$ — reaktywność reaktora,
	\item $\beta_i$ — udział neutronów opóźnionych i fotoneutronów z $i$-tej grupy w całej puli neutronów,
	\item $\Lambda$ — czas życia jednej generacji neutronów,
	\item $\lambda_i$ — stała rozpadu prekursorów neutronów opóźnionych i fotoneutronów z $i$-tej grupy (w~s$^{-1}$).
\end{itemize}

Rozważamy reaktor pracujący do chwili $t=0$ w stanie krytycznym (stacjonarnym), a w chwili $t=0$ wprowadzamy skokowo stałą reaktywność $\rho_0$.
Wobec tego w rozważanym przedziale czasowym przyjmujemy $\rho(t)\equiv \rho_0=\text{const}$, a parametry kinetyczne $(\beta_i,\lambda_i,\Lambda)$ są stałe (charakterystyczne dla danego reaktora).
Odpowiada to szczególnemu przypadkowi macierzowego opisu z poprzedniego rozdziału, w którym:\begin{equation}
	\frac{\mathrm{d}}{\mathrm{d}t} \left[\begin{array}{c}
		n(t)\\
		\uu{C_1(t)}\\
		\uu{ C_2(t)}\\
		\uu{\vdots}\\
		\uu{C_j(t)}
	\end{array}\right]= 
	\underbrace{
		\left[\begin{array}{ccccc}
			\frac{\rho_0-\beta}{\Lambda}&\lambda_1&\lambda_2&\hdots&\lambda_j\\
			\uu{\frac{\beta_1}{\Lambda}}&-\lambda_1&&\text{\huge0}&\\
			\uu{\frac{\beta_2}{\Lambda}}&&-\lambda_2&&\\
			\uu{\vdots}&\text{\huge0}&&\ddots&\\
			\uu{\frac{\beta_j}{\Lambda}}&&&&\lambda_j
		\end{array}\right]
	}_{\ \Resize{0.5cm}{\mathbb{A}}}
	\underbrace{
		\left[\begin{array}{c}
			n(t)\\
			\uu{C_1(t)}\\
			\uu{ C_2(t)}\\
			\uu{\vdots}\\
			\uu{C_j(t)}
		\end{array}\right]
	}_{\ \Resize{0.8cm}{\vec{f}(t)}}
	\label{maciero}
\end{equation}
\noindent  Macierz $\mathbb{A}$ jest kwadratowa i jej wymiar to $(j+1)\times (j+1)$, gdzie $j$ jest liczbą grup neutronów opóźnionych i fotoneutronów. Macierz $\mathbb{A}$ jest także macierzą \textbf{stałą w czasie}.  Oznacza to, że równanie (\ref{maciero}) jest \textbf{jednorodnym} równaniem różniczkowym pierwszego rzędu, typu:


\begin{equation}
	\frac{\mathrm{d}\vec{f}(t)}{\mathrm{d}t}=\mathbb{A}\vec{f}(t)
	\label{jedno}
\end{equation}

\noindent którego rozwiązaniem jest:

\begin{equation}
	\vec{f}(t)=\exp(\mathbb{A}t)\vec{f}(0)
	\label{jednorozw}
\end{equation}

\noindent gdzie:

\begin{equation}
	\vec{f}(0)=\left[\begin{array}{c}
		n(0)\\
		\uu{C_1(0)}\\
		\uu{ C_2(0)}\\
		\uu{\vdots}\\
		\uu{C_j(0)}
	\end{array}\right]
	=\left[\begin{array}{c}
		\uu{n_0}\\
		\uu{C_{1,0}}\\
		\uu{ C_{2,0}}\\
		\uu{\vdots}\\
		\uu{C_{j,0}}
	\end{array}\right]
	\label{fzero}
\end{equation}

\noindent Metoda otrzymania rozwiązania ogólnego równania (\ref{maciero}) przeanalizowana zostanie poniżej w  (podrozdział \ref{inhour}).


\subsection{Wyprowadzenie równania odwrotnych godzin}\label{inhour}
 Podanie ogólnego rozwiązania równania (\ref{maciero}), które opisane jest wzorem (\ref{jednorozw}) sprowadza się do obliczenia wyrażenia $\exp(\mathbb{A}t)$. Należy tu skorzystać z kilku własności algebry macierzy.

\begin{enumerate}[label=\Roman*.]
	\item Jeżeli macierz $\mathbb{B}$ jest diagonalna, to:
	
	\begin{equation}
		\exp(\mathbb{B})=\exp
		\left[\begin{array}{ccc}
			\uu{b_{1,1}}&&\\
			&\ddots&\text{\huge0}\\
			\text{\huge0}&&\uu{b_{i,i}}
		\end{array}\right]=
		\left[\begin{array}{ccc}
			\uu{\exp(b_{1,1})}&&\\
			&\ddots&\text{\huge0}\\
			\text{\huge0}&&\uu{\exp(b_{i,i})}
		\end{array}\right]
		\label{expM}
	\end{equation}
	
	\item Jeżeli macierz jest diagonalna, to wartości na diagonali są również wartościami własnymi tej macierzy.
	
	\item Jeżeli macierz $\mathbb{A}$ jest niediagonalna, ale diagonalizowalna, to można ją zdiagonalizować i otrzymać diagonalną macierz $\mathbb{B}$:
	
	\begin{equation}
		\mathbb{B} = \mathbb{S}^{-1}\mathbb{A} \ \mathbb{S}
		\label{MN}
	\end{equation}
	
	gdzie macierz $\mathbb{S}$ jest macierzą zmiany bazy.
	
	Diagonalizacja macierzy odbywa się poprzez odpowiednie obracanie lub/i skalowanie jej elementów. W przypadku rozwiązywania układu równań należy unikać skalowania, ponieważ zmieniłoby ono sens początkowych równań. Metodą na zdiagonalizowanie macierzy jedynie za pomocą obrotów jest \textbf{diagonalizacja metodą Jordana}. Dowód na to, że rozważana macierz $\mathbb{A}$ jest diagonalizowalna zostanie przedstawiony w rozdziale \ref{jednogrupowy} i \ref{wielogrupowe}.
	
	\item Jeżeli $\mathbb{A} = \mathbb{S}\mathbb{B} \mathbb{S}^{-1}$, to:
	
	\begin{equation}
		\mathbb{A}^n = 
		\underbrace{
			\mathbb{S}\mathbb{B}\cancel{\mathbb{S}}^{-1}\cdot\cancel{\mathbb{S}}\mathbb{B}\cancel{\mathbb{S}}^{-1}\cdot\cdot\cdot \cancel{\mathbb{S}}\mathbb{B}\mathbb{S}^{-1}
		}_{n \ \mathrm{razy}}
		= \mathbb{S}\mathbb{B}^n\mathbb{S}^{-1}
		\label{NN}
	\end{equation}
	
	\item Ponieważ funkcję $exp$ można rozwinąć w szereg Taylora, czyli w szereg wielomianowy: $\exp(x)=\sum\limits_{n=1}^{\infty}\frac{x^n}{n!}$, to korzystając z własności (\ref{NN}) otrzymujemy:
	
	\begin{equation}
		\exp(\mathbb{A}t)=
		\sum\limits_{n=1}^{\infty}\frac{(\mathbb{A}t)^n}{n!}=
		\sum\limits_{n=1}^{\infty}\frac{\mathbb{S}(\mathbb{B}t)^n\mathbb{S}^{-1}}{n!}=
		\mathbb{S}\left(\sum\limits_{n=1}^{\infty}\frac{(\mathbb{B}t)^n}{n!}\right)\mathbb{S}^{-1}=
		\mathbb{S} \exp(\mathbb{B}t)\mathbb{S}^{-1}
		\label{wielo}
	\end{equation}
	
\end{enumerate}

\noindent Korzystając z wymienionych powyżej własności I--V, układ równań (\ref{maciero}) można rozwiązać w~następujący sposób:

\begin{enumerate}
	
	\item Zapisać układ równań kinetyki w postaci macierzowej. Upewnić się, że macierz współczynników (macierz $\mathbb{A}$) jest niezależna od czasu.
	
	\item Zdiagonalizować macierz współczynników (macierz $\mathbb{A}$) metodą Jordana i otrzymać diagonalną macierz $\mathbb{B}$.
	
	\item Otrzymać rozwiązanie ogólne układu równań, którym będzie:
\end{enumerate}
	\begin{equation}
		\vec{f}(t)=\mathbb{S}\exp(\mathbb{B}t)\mathbb{S}^{-1}\vec{f}(0)
		\label{rozwogolne}
	\end{equation}
	\textbf{Jeżeli macierz jest kwadratowa o wymiarze $n\times n$ i posiada $n$ różnych wartości własnych, to jest diagonalizowalna.}
	Poniżej wykażemy, że analizowana macierz kwadratowa $\mathbb{A}$ o wymiarze \mbox{$(j+1)\times (j+1)$} spełnia powyższy warunek.
	
	Dla uproszczenia zapisu rozważaną macierz $\mathbb{A}$ przedstawimy za pomocą elementów macierzy oznaczonych małymi literkami $a$ z odpowiednimi indeksami w następujący sposób:
	\begin{equation}
		\mathbb{A}= 
		\left[\begin{array}{ccccc}
			\frac{\rho_0-\beta}{\Lambda}&\lambda_1&\lambda_2&\hdots&\lambda_j\\
			\uu{\frac{\beta_1}{\Lambda}}&-\lambda_1&&\text{\huge0}&\\
			\uu{\frac{\beta_2}{\Lambda}}&&-\lambda_2&&\\
			\uu{\vdots}&\text{\huge0}&&\ddots&\\
			\uu{\frac{\beta_j}{\Lambda}}&&&&-\lambda_j
		\end{array}\right]
		=
		\left[\begin{array}{ccccc}
			\uu{a_{1,1}}&a_{1,2}&a_{1,3}&\hdots&a_{1,j+1}\\
			\uu{a_{2,1}}&a_{2,2}&&\text{\huge0}&\\
			\uu{a_{3,1}}&&a_{3,3}&&\\
			\uu{\vdots}&\text{\huge0}&&\ddots&\\
			\uu{a_{j+1,1}}&&&&a_{j+1,j+1}
		\end{array}\right]
		\label{macierouuu}
	\end{equation}
	
	%\noindent Ze względu na kształt macierzy $\mathbb{A}$, wiadomo, że poszczególne wartości $\lambda_i$ będą różne, ponieważ reprezentują stałe rozpadu, charakteryzujące różne grupy neutronów opóźnionych i fotoneutronów. Różna od nich będzie także wartość $\frac{\rho_0-\beta}{\Lambda}$. Oznacza to, że macierz $\mathbb{A}$ będzie posiadała $j+1$ różnych od siebie, wyrazów $a_{i,i}$ na diagonali.
	%\noindent Poniżej dowiedziemy, że macierz  $\mathbb{A}$ posiada $(j+1)$ różnych wartości własnych. 
	
	\noindent Jeżeli wektor $\vec{v}_k$ jest \textbf{wektorem własnym} macierzy $\mathbb{A}$, to zgodnie z jego definicją jest \textbf{niezerowy} i spełnia następujące równanie: 
	
	\begin{equation}
		\mathbb{A}\cdot \vec{v}_k = \alpha_k \cdot \vec{v}_k 
		\label{czczczczcz}
	\end{equation}
	
	\noindent gdzie $\alpha_k$ jest wartością własną, stowarzyszoną z tym wektorem własnym.
	
	Jeżeli zdefiniujemy macierz $\mathbb{M}(\alpha_k)$, jako:
	
	\begin{equation}
		\mathbb{M}(\alpha_k)= 
		\mathbb{A}-\alpha_k\cdot \mathbb{1}
		\label{yuyuyuyuyuy}
	\end{equation}
	
	\noindent gdzie macierz $\mathbb{1}$ jest macierzą jednostkową, to:
	
	\begin{equation}
		\mathbb{M}(\alpha_k)\cdot \vec{v}_k= 0
		\label{brbrbrbrb}
	\end{equation}
	
	\noindent ponieważ:
	
	\begin{equation}
		(\mathbb{A}-\alpha_k\cdot \mathbb{1})\cdot \vec{v}_k=\mathbb{A}\cdot \vec{v}_k -\alpha_k \cdot \vec{v}_k =  0
		\label{ioioioioi}
	\end{equation}
	
	\noindent Gdyby macierz $\mathbb{M}$ była macierzą odwracalną, to istniałaby macierz do niej odwrotna, czyli $\mathbb{M}^{-1}$. Wtedy, mnożąc równanie  (\ref{brbrbrbrb}) obustronnie przez macierz $\mathbb{M}^{-1}$ otrzymalibyśmy:
	
	\begin{equation}
		\underbrace{
			\mathbb{M}^{-1}\cdot \mathbb{M}
		}_{\ \Resize{0.25cm}{\mathbb{1}}}
		\cdot \vec{v}_k =
		\underbrace{
			\mathbb{M}^{-1} \cdot 0
		}_{\ \Resize{0.2cm}{0}}
		\label{udtwwjdu}
	\end{equation}
	
	\noindent a powyższe równanie (\ref{udtwwjdu}) byłoby spełnione jedynie wtedy, gdyby wektor $\vec{v}_k$ był zerowy, czyli $\vec{v}_k =0$. A to jest sprzeczne z definicją wektora własnego, który musi być niezerowy. Oznacza to, że macierz $\mathbb{M}$ \textbf{nie jest macierzą odwracalną}. A \textbf{wyznacznik każdej macierzy nieodwracalnej jest równy zero}. Tak więc:
	
	\begin{equation}
		\det \mathbb{M}(\alpha_k)=
		\det
		(\mathbb{A}-\alpha_k \cdot \mathbb{1}) 
		= 0
		\label{ertyerty}
	\end{equation}
	
	\noindent Wobec powyższego wartości własne macierzy $\mathbb{A}$, oznaczone jako $\alpha_k$, spełniają równanie (\ref{ertyerty}).
	
	
	Definiujemy macierz $\mathbb{M}(\alpha)$, jako funkcję ciągłego parametru $\alpha$:
	
	\begin{equation}
		\mathbb{M}(\alpha)= 
		\mathbb{A}-\alpha\cdot \mathbb{1}
		=
		\left[\begin{array}{ccccc}
			\uu{a_{1,1}-\alpha}&a_{1,2}&a_{1,3}&\hdots&a_{1,j+1}\\
			\uu{a_{2,1}}&a_{2,2}-\alpha&&\text{\huge0}&\\
			\uu{a_{3,1}}&&a_{3,3}-\alpha&&\\
			\uu{\vdots}&\text{\huge0}&&\ddots&\\
			\uu{a_{j+1,1}}&&&&a_{j+1,j+1}-\alpha
		\end{array}\right]
		\label{mmmmmmmmm}
	\end{equation}
	Wyznacznik tej macierzy, $\det(\mathbb{M}(\alpha))$, będzie funkcją wielomianową parametru $\alpha$. Zgodnie z~równaniem (\ref{ertyerty}), wartości własne $\alpha_k$ będą miejscami zerowymi (pierwiastkami) tego wielomianu, czyli:
	
	\begin{equation}
		\det
		(\mathbb{M}(\alpha_k)) 
		= 0
		\label{wewewewewewe}
	\end{equation}
	
	\subsection{Przypadek jednogrupowy, $j=1$} \label{jednogrupowy}
	
	W przypadku jednej grupy neutronów ($j=1$) wzór (\ref{wewewewewewe}) przybierze postać:
	\begin{equation}
		\det(\mathbb{M}(\alpha)) =
		\left|\begin{array}{cc}
			\frac{\rho_0-\beta}{\Lambda} - \alpha&\lambda_1\\
			\uu{\frac{\beta}{\Lambda}}&-\lambda_1 - \alpha
		\end{array}\right|
		= (\frac{\rho_0-\beta}{\Lambda} - \alpha)(-\lambda_1 - \alpha) - \frac{\beta}{\Lambda}\lambda_1
		\label{babel}
	\end{equation}
	
	\noindent Podstawiając równanie (\ref{babel}) do równania (\ref{wewewewewewe}) i rozwiązując je (czyli znajdując jego miejsca zerowe), otrzymamy dwie różne wartości własne: 
	
	\begin{equation}
		\alpha_{\pm}=-\frac{\Lambda\lambda-\rho_0+\beta}{2\Lambda}\pm\frac{\sqrt{\Lambda^2\lambda^2+2\Lambda\lambda(\rho_0+\beta)+\rho_0^2+\beta^2-2\rho_0\beta}}{2\Lambda}
		\label{apm}
	\end{equation}
	
	\noindent Ponieważ w tym przypadku (jednogrupowym) macierz $\mathbb{A}$ o wymiarze $2\times 2$ posiada dwie różne wartości własne, co oznacza, że \textbf{jest macierzą diagonalizowalną}.
	
	W przypadku jednogrupowym, który jest stosunkowo prosty do przeprowadzenia dokładnego, analitycznego rozwiązania, macierz diagonalna $\mathbb{B}$ będzie równa:
	
	\begin{equation}
		\mathbb{B}=\left[\begin{array}{cc}
			\alpha_+&0\\
			\uu{0}&\alpha_-
		\end{array}\right]
		\label{bee}
	\end{equation}
	
	\noindent Następnie należy znaleźć macierz obrotu $\mathbb{S}$. Będzie złożona z wektorów własnych $\vec{\mathfrak{s}}$ macierzy $\mathbb{A}$, spełniających równanie:
	
	\begin{equation}
		\mathbb{A}\vec{\mathfrak{s}}_{\pm}=\alpha_{\pm}\vec{\mathfrak{s}}_{\pm}
		\label{vecss}
	\end{equation}
	
	\noindent Wtedy macierz obrotu S będzie równa:
	
	\begin{equation}
		\mathbb{S}=[\vec{\mathfrak{s}}_+,\vec{\mathfrak{s}}_-]=
		\left[\begin{array}{cc}
			\frac{\Lambda}{\beta}(\lambda+\alpha_+)&\frac{\Lambda}{\beta}(\lambda+\alpha_-)\\
			\uu{1}&1
		\end{array}\right]=
		\left[\begin{array}{cc}
			x_+&x_-\\
			\uu{1}&1
		\end{array}\right]
		\label{svec}
	\end{equation}
	
	\noindent gdzie dla uproszczenia zapisu przyjęto, że:
	
	\begin{equation}
		x_{\pm}=\frac{\Lambda}{\beta}(\lambda+\alpha_{\pm})
		\label{ixsde}
	\end{equation}
	
	\noindent W kolejnym kroku należy obliczyć macierz odwrotną $\mathbb{S}^{-1}$. W tym przypadku będzie ona równa:
	
	\begin{equation}
		\mathbb{S}^{-1}=\frac{1}{\det[\mathbb{S}]}
		\left[\begin{array}{cc}
			\uu{1}&-\frac{\Lambda}{\beta}(\lambda+\alpha_-)\\
			\uu{-1}&\frac{\Lambda}{\beta}(\lambda+\alpha_+)
		\end{array}\right]=\frac{1}{x_+-x_-}
		\left[\begin{array}{cc}
			\uu{1}&-x_-\\
			\uu{-1}&x_+
		\end{array}\right]
		\label{s-1}
	\end{equation}
	
	\noindent Zgodnie z (\ref{expM}) otrzymujemy, że:
	
	\begin{equation}
		\exp(\mathbb{B}t)=
		\left[\begin{array}{cc}
			\exp(\alpha_+t)&0\\
			\uu{0}&exp(\alpha_-t)
		\end{array}\right]=
		\left[\begin{array}{cc}
			\mathfrak{e}_+&0\\
			\uu{0}&\mathfrak{e}_-
		\end{array}\right]
		\label{expBe}
	\end{equation}
	
	\noindent gdzie dla uproszczenia zapisu przyjęto, że:
	
	\begin{equation}
		\mathfrak{e}_{\pm}=\exp(\alpha_{\pm}t)
		\label{eee}
	\end{equation}
	
	\noindent Aby otrzymać rozwiązanie ogólne rozważanego jednogrupowego przypadku, należy skorzystać ze wzoru (\ref{rozwogolne}):
	
	\begin{equation}
		\begin{aligned}
			\left[\begin{array}{c}
				n(t)\\
				\uu{C(t)}
			\end{array}\right]=&\frac{1}{x_+-x_-}
			\left[\begin{array}{cc}
				x_+&x_-\\
				\uu{1}&1
			\end{array}\right]
			\left[\begin{array}{cc}
				\mathfrak{e}_+&0\\
				\uu{0}&\mathfrak{e}_-
			\end{array}\right]
			\left[\begin{array}{cc}
				1&-x_-\\
				\uu{-1}&x_+
			\end{array}\right]
			\left[\begin{array}{c}
				n_0\\
				\uu{C_0}
			\end{array}\right]=\\
			&\\
			&=\frac{1}{x_+-x_-}
			\left[\begin{array}{cc}
				x_+\mathfrak{e}_+-x_-\mathfrak{e}_-&x_+x_-(\mathfrak{e}_--\mathfrak{e}_+)\\
				\uu{\mathfrak{e}_+-\mathfrak{e}_-}&x_+\mathfrak{e}_--x_-\mathfrak{e}_+
			\end{array}\right]
			\left[\begin{array}{c}
				n_0\\
				\uu{C_0}
			\end{array}\right]
			\label{jednorozwog}
		\end{aligned}
	\end{equation}
	
	\noindent Jak widać z (\ref{jednorozwog})rozwiązanie ogólne prostego, jednogrupowego przypadku jest skomplikowane. Ale obserwując jego kształt, można zauważyć, że w tym przypadku, gdy mamy dwa równania kinetyki (tzn. wymiar macierzy $\mathbb{A}$ równy jest 2), to w rozwiązaniach ogólnych wystąpią wyrażenia liniowe dwóch różnych eksponentów. Wykorzystując fakt, że zarówno iloczyn, jak i iloraz eksponentów również jest eksponentem, liniowe wyrażenia eksponentów można przekształcić do odpowiedniej sumy eksponentów o odpowiednich wykładnikach. To oznacza, że wzór (\ref{jednorozwog})można uprościć do postaci:
	
	\begin{equation}
		n(t)=N_+e^{\alpha_+t}+N_-e^{\alpha_-t}
		\label{n_suma_exp}
	\end{equation}
	
	\begin{equation}
		C(t)=C_{+}e^{\alpha_+t}+C_{-}e^{\alpha_-t}
		\label{C_suma_exp}
	\end{equation}
	
	\noindent gdzie:
	
	\begin{equation}
		N_{\pm}=\pm \frac{n_0x_{\pm}-C_0x_+x_-}{x_+-x_-}
		\label{nenenenenenenenne}
	\end{equation}
	
	\noindent oraz 
	
	\begin{equation}
		C_{\pm}=\pm \frac{n_0-C_0x_{\mp}}{x_+-x_-}
		\label{yuertueyt}
	\end{equation}
\newpage
\subsection{Rozwiązanie wielogrupowe, $j=6$}\label{wielogrupowe}
W sytuacji kiedy chcemy rozdzielić neutrony opóźnione na sześć różnych grup, musimy  rozwiązać  układ kinetyki składający się aż z  siedmiu równań, które przy rozwiązywaniu analitycznym obarcza nas niepotrzebnym chaosem. 
Dlatego też dla przypadku sześciogrupowego  $\mathbb{M}$ skorzystamy z obliczeń numerycznych, w którym niepodzielnie rządzi porządek i precyzja. 


Podział grupowy neutronów opóźnionych wraz ze stałymi zaniku ich prekursorów oraz odpowiadającymi im czasami połowicznego zaniku i udziałami poszczególnych grup w~bilansie neutronów reaktora MARIA \cite{kubowski}, przedstawiono w~Tabeli~\ref{tab2}. Wartości te zostały użyte do obliczenia wyznacznika macierzy $\mathbb{M}$. Na Rys. \ref{iretwor} przedstawiono obliczony numerycznie wyznacznik macierzy $\mathbb{M}$, jako funkcję parametru $\alpha$ dla trzech wartości wprowadzonej do rdzenia reaktywności, wynoszących, odpowiednio, 0,5 \$, 1 \$ oraz 1,5 \$. Miejsca zerowe oznaczono punktami, a ich wartości przedstawiono w Tabeli \ref{tab1}. Jako czas życia jednej generacji neutronów, przyjęto $\Lambda = 146 \ \mu \mathrm{s}$.

 
\begin{table}[H]
	\centering
	\caption{Udziały, stałe zaniku oraz czasy połowicznego zaniku prekursorów neutronów opóźnionych \cite{kubowski}.}
	\label{tab2}
	\renewcommand{\arraystretch}{1.2}
	\begin{tabular}{|c|c|c|c|c|}
		\hline
		Przypadek & Nr grupy $i$ & $\beta_i$ & $\lambda_i$, $\mathrm{s^{-1}}$ & $T_{1/2}$, s \\
		\hline
		\multirow{6}{*}{Wielogrupowy}
		& 1 & $2{,}43 \cdot 10^{-4}$ & $1{,}27 \cdot 10^{-2}$ & 54{,}6 \\
		& 2 & $1{,}363 \cdot 10^{-3}$ & $3{,}17 \cdot 10^{-2}$ & 21{,}9 \\
		& 3 & $1{,}203 \cdot 10^{-3}$ & $1{,}15 \cdot 10^{-1}$ & 6{,}03 \\
		& 4 & $2{,}605 \cdot 10^{-3}$ & $3{,}11 \cdot 10^{-1}$ & 2{,}20 \\
		& 5 & $8{,}19 \cdot 10^{-4}$ & $1{,}40$ & 0{,}50 \\
		& 6 & $1{,}67 \cdot 10^{-4}$ & $3{,}87$ & 0{,}18 \\
		\hline
		Jednogrupowy
		& \multicolumn{2}{c|}{$\beta = \sum\limits_{i=1}^{6} \beta_i = 6{,}4 \cdot 10^{-3}$}
		& \multicolumn{2}{c|}{$\hat{\lambda} = 0{,}44~\mathrm{s^{-1}}$} \\
		\hline
	\end{tabular}
\end{table}



\begin{table}[H]
	\centering
	\caption{Miejsca zerowe wyznacznika $\det\mathbb{M}(\alpha_i)$ dla wybranych wartości wprowadzonej reaktywności.}
	\label{tab1}
	\renewcommand{\arraystretch}{1.2}
	\begin{tabular}{|c|c|c|c|}
		\hline
		\multirow{2}{*}{$\alpha_i$} &
		\multicolumn{3}{c|}{Reaktywność $\rho$} \\ \cline{2-4}
		& $\rho = 0{,}5~\$$ & $\rho = 1{,}0~\$$ & $\rho = 1{,}5~\$$ \\
	\hline
	$\alpha_1$ & 0,180  & 3,729  & 22,707 \\
	$\alpha_2$ & -0,013 & -0,013 & -0,013 \\
	$\alpha_3$ & -0,047 & -0,039 & -0,037 \\
	$\alpha_4$ & -0,161 & -0,140 & -0,132 \\
	$\alpha_5$ & -1,113 & -0,756 & -0,500 \\
	$\alpha_6$ & -3,683 & -2,936 & -1,770 \\
	$\alpha_7$ & -22,820& -5,634 & -4,078 \\
	\hline
	
	\end{tabular}
\end{table}

\begin{figure}[H]
	\centering
	\includegraphics[width= 16cm]{figures/M(alfa)_rho_all.png}
	\caption{Obliczona wartość wyznacznika macierzy $\mathbb{M}$, jako funkcja parametru $\alpha$ dla trzech wartości wprowadzonej do rdzenia reaktywności, wynoszących, odpowiednio, 0,5 \$, 1 \$ oraz 1,5 \$. Miejsca zerowe wyznacznika zaznaczono na rysunku za pomocą punktów. Poszczególne kolory odpowiadają wartościom wprowadzonej reaktywności w \$.}\label{iretwor}
\end{figure}

Jak widać wyznacznik macierzy $\mathbb{M}$ dla przypadku sześciogrupowego posiada siedem różnych miejsc zerowych, będących jednocześnie siedmioma różnymi wartościami własnymi macierzy $\mathbb{A}$, więc macierz $\mathbb{A}$ \textbf{jest macierzą diagonalizowalną}. I analogicznie do przypadku jednogrupowego, rozwiązania równań kinetyki punktowej (\ref{maciero})będą liniowymi kombinacjami siedmiu eksponentów, które można przekształcić do postaci sumy siedmiu eksponentów z odpowiednimi współczynnikami:

\begin{equation}
	n(t)=\sum_{k=1}^7 N_ke^{\alpha_kt}
	\label{werty}
\end{equation}

\begin{equation}
	C_i(t)=\sum_{k=1}^7 C_{k,i}e^{\alpha_kt}
	\label{dfghj}
\end{equation}

\noindent ze względu na skomplikowaną postać rozwiązań równań kinetyki punktowej w przypadku sześciu grup neutronów opóźnionych analityczne postaci współczynników $N_k$ oraz $C_{k,i}$ nie zostaną podane. 

Warto także podkreślić, że wśród otrzymanych wartości własnych, zawsze tylko jedna wartość własna jest dodatnia, a pozostałe są ujemne. Oznacza to, że dla przypadku każdej z trzech zastosowanych reaktywności tylko eksponent z dodatnią wartością własną będzie "wybuchać" w nieskończoności, natomiast wszystkie pozostałe (z wartościami ujemnymi) zanikną.
	
\subsection{Rozwiązanie asymptotyczne równania kinetyki}
Porównując rozwiązania równań kinetyki punktowej dla przypadku jednogrupowego 
i sześciogrupowego, można zauważyć, że przybierają one kształt:

\begin{equation}
n(t)=\sum\limits_{k=1}^{j+1}N_ke^{s_kt}
\label{n_suma_exp}
\end{equation}

\begin{equation}
C_i(t)=\sum\limits_{k=1}^{j+1}C_{k,i}e^{s_kt}
\label{C_suma_exp}
\end{equation}

\noindent gdzie $N_k$ oraz $C_{k,i}$ są stałymi, a $s_k$ mają wymiar $\frac{1}{\mathrm{czas}}$ i są wartościami własnymi macierzy $\mathbb{A}$ [\ref{macierouuu}].

Ze względu na kształt zależności \ref{n_suma_exp}, \ref{C_suma_exp} można przewidzieć, jak $n(t)$ oraz $C_i(t)$ będą się zachowywały przy $t\rightarrow\infty$. Mianowicie człon z eksponentem o największym wykładniku (największym wykładniku, $s_m=max(s_k)$, ze wszystkich występujących $s_k$), najszybciej rosnąc, zdominuje wszystkie pozostałe. Można to pokazać, zauważając, że~iloraz dwóch eksponentów:

\begin{equation}
\frac{A_xe^{s_xt}}{A_ye^{s_yt}}=\frac{A_x}{A_y}e^{(s_x-s_y)t}\xrightarrow[t\rightarrow\infty]{}
\begin{cases}
    \infty,& s_x>s_y\\
    \sfrac{A_x}{A_y},& s_x=s_y \\
    0,              & s_x<s_y
\end{cases}
\label{AdoA}
\end{equation}

\noindent gdzie $A_x$ oraz $A_y$ są stałymi.

Wtedy, korzystając z własności (\ref{AdoA}) oraz definiując $s_m=max(s_k)$, widać, że:

\begin{equation}
\sum\limits_{k=1}^{j+1}\frac{A_ke^{s_kt}}{A_me^{s_mt}}\xrightarrow[t\rightarrow\infty]{ }1
\label{tralala}
\end{equation}

\noindent Czyli:

\begin{equation}
\sum\limits_{k=1}^{j+1}A_ke^{s_kt}\xrightarrow[t\rightarrow\infty]{ } A_me^{s_mt}.
\label{tralala2}
\end{equation}

\noindent To powoduje, że ostatecznie, po odpowiednio długim czasie wartość $n(t)$ zbiegnie asymptotycznie do postaci:

\begin{equation}
n(t)\xrightarrow[t\rightarrow\infty]{}N_me^{s_mt}=N_me^{\frac{t}{T}},
\label{rtyurtuturtyrtrtu}
\end{equation}

\noindent oraz wartość $C_i(t)$ zbiegnie asymptotycznie do postaci:

\begin{equation}
C_i(t)\xrightarrow[t\rightarrow\infty]{}C_{m,i}e^{s_mt}=C_{m,i}e^{\frac{t}{T}},
\label{qerqwetqreyqery}
\end{equation}

\noindent gdzie $\frac{1}{T}=s_m$. $T$ jest z definicji okresem reaktora, który się obserwuje (wszystkie pozostałe człony $e^{s_it}$ zostały zdominowane przez człon $e^{s_mt}=e^{\frac{t}{T}}$). Przykładowe zachowanie się mocy reaktora po wprowadzeniu do rdzenia stałej porcji reaktywności i zaobserwowanie okresu reaktora przedstawione zostało na Rysunku \ref{T}. Są to wyniki obliczeń (numerycznych rozwiązań równań kinetyki punktowej.) dla przypadku sześciogrupowego (Rozdział \ref{wielogrupowe}). Na rysunku tym widać, że po upływie około 13,13 sekund po wprowadzeniu dodatniej reaktywności, wzrost mocy staje się funkcją wykładniczą (linia prosta na wykresie, gdy oś pionowej współrzędnej jest w~skali logarytmicznej).

\begin{figure}[H]
	\centering
		\includegraphics[angle=0, width= 14cm]{figures/przebieg_mocy_rho_0_5v2.png}
	\caption{Przebieg mocy reaktora w czasie. Wartość wprowadzonej reaktywności (wynosiła +0,5 $\$$) \cite{Tadi}.}\label{T}
\end{figure}

\begin{figure}[H]
	\centering
	\includegraphics[angle=0, width= 14cm]{figures/przebieg_mocy_rho_1.png}
	\caption{Przebieg mocy reaktora w czasie. Wartość wprowadzonej reaktywności (wynosiła +1 $\$$) \cite{Tadi}.}\label{T2}
\end{figure}
\begin{figure}[H]
	\centering
	\includegraphics[angle=0, width= 14cm]{figures/przebieg_mocy_rho_1_5.png}
	\caption{Przebieg mocy reaktora w czasie. Wartości wprowadzonej reaktywności (+1,5 $\$$)   \cite{Tadi}.}\label{T3}
\end{figure}


Kształt rozwiązań (\ref{rtyurtuturtyrtrtu}) oraz (\ref{qerqwetqreyqery}) można podstawić do układu równań kinetyki reaktora punktowego (\ref{maciero}).
\subsection{Podstawianie rozwiązania}}
Podstawiając asymptotyczne rozwiązania funkcji $n(t)$ oraz $C_i(t)$ ze wzorów (\ref{rtyurtuturtyrtrtu}) i (\ref{qerqwetqreyqery}) do równań kinetyki punktowej (\ref{nodt}) oraz (\ref{ciodt}) otrzyma się, że:

\begin{equation}
	\frac{\mathrm{d}}{\mathrm{d}t}(N_me^{\frac{t}{T}})=\frac{\rho_0-\beta}{\Lambda}N_me^{\frac{t}{T}}+\sum_{i=1}^{j} \lambda_iC_{m,i}e^{\frac{t}{T}} 
	\label{nodtabababab}
\end{equation}

\begin{equation}
	\frac{\mathrm{d}}{\mathrm{d}t}(C_{m,i}e^{\frac{t}{T}})=\frac{\beta_i}{\Lambda}N_me^{\frac{t}{T}}-\lambda_iC_{m,i}e^{\frac{t}{T}}
	\label{ciodtababababab}
\end{equation}

\noindent Różniczkując lewą stronę tych równań, otrzyma się:

\begin{equation}
	\frac{N_m}{T}\bcancel{e^{\frac{t}{T}}}=\frac{\rho_0-\beta}{\Lambda}N_m\bcancel{e^{\frac{t}{T}}}+\sum_{i=1}^{j} \lambda_iC_{m,i}\bcancel{e^{\frac{t}{T}}}
	\label{nodbab}
\end{equation}

\begin{equation}
	\frac{C_{m,i}}{T}\bcancel{e^{\frac{t}{T}}}=\frac{\beta_i}{\Lambda}N_m\bcancel{e^{\frac{t}{T}}}-\lambda_iC_{m,i}\bcancel{e^{\frac{t}{T}}}
	\label{ciodtbabab}
\end{equation}

\noindent czyli:

\begin{equation}
	\frac{N_m}{T}=\frac{\rho_0-\beta}{\Lambda}N_m+\sum_{i=1}^{j} \lambda_iC_{m,i}
	\label{nodbaqqqqqqb}
\end{equation}

\begin{equation}
	\frac{C_{m,i}}{T}=\frac{\beta_i}{\Lambda}N_m-\lambda_iC_{m,i}
	\label{ciodtbabqqqqqqab}
\end{equation}

\noindent Wyznaczając $C_{m,i}$ ze wzoru (\ref{ciodtbabqqqqqqab}) i podstawiając do wzoru (\ref{nodbaqqqqqqb}) otrzymamy, że

\begin{equation}
	\frac{\bcancel{N_m}}{T}=\frac{\rho_0-\beta}{\Lambda}\bcancel{N_m}+\frac{\bcancel{N_m}}{\Lambda}\sum_{i=1}^{j} \frac{\beta_i}{\frac{1}{T}+\lambda_i}
	\label{nodbwewewrwtwyaqqqqqqb}
\end{equation}

\noindent Przekształcając powyższe równanie można otrzymać wzór na wprowadzoną reaktywność $\rho_0$:

\begin{equation}
	\rho_0=\frac{\Lambda}{T}+\beta - \sum_{i=1}^{j} \frac{\lambda_i\beta_i}{\frac{1}{T}+\lambda_i}
	\label{rrrrrrrrrrryyyy}
\end{equation}

\noindent i pamiętając, że $\beta=\sum_{i=1}^{j}\beta_i$ można wciągnąć $\beta$ pod sumę we wzorze (\ref{rrrrrrrrrrryyyy}) i po uporządkowaniu ostatecznie otrzymać \textbf{wzór odwrotnych godzin}:

\begin{equation}\label{inhour}
	\rho_0=\frac{1}{T}\left(\Lambda+ \sum_{i=1}^{j} \frac{\beta_i}{\frac{1}{T}+\lambda_i}\right)
	\label{wzodgo}
\end{equation}

\noindent Jeżeli wprowadzaną reaktywność chcemy wyrazić w jednostkach $\$$, to, należy pamiętać, że $\rho_{\$}=\frac{\rho}{\beta}$, więc wtedy wzór odwrotnych godzin przyjmie postać:

\begin{equation}
	\rho_{\$}=\frac{1}{T\beta}\left(\Lambda+ \sum_{i=1}^{j} \frac{\beta_i}{\frac{1}{T}+\lambda_i}\right)
	\label{wzodgoSS}
\end{equation}

\noindent Wykres zależności (\ref{wzodgoSS}) przedstawiono na Rysunku \ref{umpaumpa}.

\begin{figure}[H]
	\centering
	\scalebox{1}[1]{\includegraphics[width= 14cm]{figures/Okres reaaktora.png}}
	\caption{Zależność reaktywności wprowadzanej do rdzenia $\rho_{\$}$ od okresu reaktora $T$. Przypadek jednogrupowy i sześciogrupowy.	}\label{umpaumpa}
\end{figure}
Widać, że przypadek jednogrupowy jest bardziej konserwatywny i daje wyraźnie krótsze czasy okresu reaktora, niż przypadek wielogrupowy.






\section{Porównanie krzywej teoretycznej i eksperymentalnej}
\subsection{Cel pomiaru i warunki pracy reaktora}

Celem pomiaru było wyznaczenie eksperymentalnej charakterystyki efektywności pręta
kompensacyjnego PK6, tj. zależności reaktywności wnoszonej do rdzenia $\rho$ (w zapisie
dolarowym $\rho_{\text{\$}}
$) od położenia pręta $z$, a następnie porównanie jej z krzywą teoretyczną
o kształcie litery ``S'' wynikającą z przyjętego modelu rozkładu strumienia neutronów w rdzeniu.
Pomiary wykonano w stanach ustalonych reaktora, przy pracy w zakresie mocy dobranym tak,
aby nie wprowadzać istotnych błędów wynikających z efektów temperaturowych przy ograniczonym chłodzeniu. 
\subsection{Metoda zrzutu \cite{OT-12}}\label{metoda_zrzutu}
Przy pomocy metody zrzutu można przeprowadzić pomiar całkowitej reaktywności wnoszonej przez poszczególne pręty (PK, PB i PAR), ich części jak również przez grupy prętów. Opiera się ona na pomiarze zmiany strumienia neutronów w rdzeniu reaktora pod wpływem nagłego wprowadzenia ujemnej reaktywności (zrzut pręta pochłaniającego). Z przebiegu tych zmian można określić wielkość wprowadzonej reaktywności. Dokonuje się tego przy użyciu sprzętu komputerowego ze specjalistycznym oprogramowaniem. 
\subsubsection{Przebieg pomiaru}
Najważniejsze podpunkty przebiegu pomiaru: 
\begin{itemize}
	

\item Osiągnąć moc rzędu $\mathbf{30\ \text{kW}-100\ \text{kW}}$
, włączyć PAR na pracę automatyczną, ustabilizować reaktor. 
\item nadkompensować pręty PK w taki sposób, aby pręty mierzone metodą zrzutu znajdowały się na żądanej wysokości.
\item Ustawić komorę toru ILR-5 tak, aby częstość zliczeń tej linii nie przekraczała $\mathbf{5\times10^4\,\dfrac{imp}{sek}}$.
\item Przełączyć w dolne położenie dźwignie przełączników odpowiadających zrzucanym prętom.
\item Uruchomić program pomiarowy PRETY.
\item Po ok. 1 min przełączyć PAR z pracy automatycznej na ręczną i na sygnał zaprogramowanych prętów).
\item Pomiar kontynuować przez 5 do 6 min w celu zarejestrowania wpływu tego zrzutu na moc reaktora\cite{OT-12}.
\end{itemize}
\subsection{Metoda okresu \cite{OT-12}}\label{metoda_okresu}
Pomiar efektywności prętów pochłaniających metodą okresu ustalonego polega na pomiarze okresu reaktora wywołanego wprowadzeniem dodatniej reaktywności poprzez stopniowe wyciąganie pręta pochłaniającego. Stosuje się ją dla stosunkowo małych reaktywności. Aby zapewnić możliwie dużą dokładność pomiarów należy przeprowadzić je na mocy ok. $\mathbf{30\ \text{kW}}$. Moc ta jest na tyle duża aby zminimalizować wpływ źródła fotoneutronów – zaś z drugiej strony na tyle mała, aby nie wprowadzać do pomiarów błędów wynikających z efektów temperaturowych przy ograniczonym chłodzeniu. Pomiar i rejestracja reaktywności oraz okresu reaktora realizowane są \textbf{,,on line''} za pomocą komputera PC i niezależnie przez Zespół Zmianowy – na podstawie pomiaru czasu podwojenia mocy T2x. Przyrost reaktywności w zależności od czasu podwojenia oblicza się na podstawie równania odwrotnych godzin.
\subsubsection{Przebieg pomiaru}

\begin{itemize}
	\item Dokonać rozruchu reaktora zgodnie z procedurą wg Instrukcji nr \textbf{ZR-23-ZR-12}.
	\item Osiągnąć moc rzędu $\mathbf{30\ \text{kW}-100\ \text{kW}}$, włączyć PAR na pracę automatyczną, ustabilizować reaktor.
	\item Prekompensować pręty PK w taki sposób, aby mierzony pręt znajdował się w dolnym położeniu (informacja na ten temat powinna być zawarta w instrukcji jednorazowej).
	\item Przełączyć PAR z pracy automatycznej na ręczną, zmienić nastawę na Nastawniku Mocy Reaktora NMR-5 o dekadę w górę.
	\item Podnieść kalibrowany pręt o $\Delta h$ wg Instrukcji Jednorazowej.
	\item Pomiar i rejestrację reaktywności oraz okresu reaktora prowadzić \textbf{,,on line''} za pomocą komputera i niezależnie przez Zespół Zmianowy – na podstawie pomiaru czasu podwojenia mocy T2x. Pomiar czasu podwojenia (w sytuacji ustalonego okresu) należy rozpocząć po upływie kilkudziesięciu sekund (ok. $80 \text{ sekund}$) od chwili wprowadzenia zaburzenia (wprowadzenie reaktywności przez skok pręta). Pomiar czasu podwojenia należy przeprowadzać trzykrotnie (\mathbb{$10 - 20, 20 - 40, 40 - 80\%$} wg wskazań 1TPP) a do obliczeń przyjmować wartość średnią. Wyniki pomiarów wpisać do protokołu z pomiarów (załącznik do instrukcji jednorazowej)\cite{OT-12}

\end{itemize}
\subsection{Excel - zaprojektowanie kroków pomiarowych}


\subsubsection{Cel}
Celem procedury jest wyznaczenie kolejnych przemieszczeń pręta regulacyjnego (w mm) o zbliżonej wartości przyrostom reaktywności wyrażonej w dolarach.




\subsubsection{Zrzuty ekranu arkusza Excel}
Na rys.~\ref{fig:excel_frontend} pokazano arkusz operatorski (warstwa \emph{front-end}), w którym użytkownik
wprowadza wyniki pomiaru wagi pręta (w~\(\$\)) oraz liczbę kroków \(N\). Arkusz automatycznie wyznacza:
\(\,W_{\mathrm{tot}}\) (średnia z trzech pomiarów wagi pręta metodą zrzutu), skok reaktywności dla każdego z planowanych kroków pomiarowych \(\Delta\rho_\$=W_{\mathrm{tot}}/N\), szacowany okres reaktora \(T\)
oraz tabelę kolejnych zagłębień pręta dla projektowanych kroków pomiarowych: skumulowane zagłębienie \(x_k\) i~przyrost w kroku \(\Delta h_k\).
Dodatkowo przedstawiona jest krzywa charakterystyki pręta wraz z naniesionymi punktami, odpowiadającymi krokom, gdzie podczas metody zrzutu[\ref{metoda_zrzutu}] ustalono, że waga PK6 wynosi 1.72\$ co warto mieć w pamięci w podrozdziale \ref{tabele pomiarowe}.
Natomiast na  rys.~\ref{fig:excel_backend} przedstawiono arkusz obliczeniowy (warstwa \emph{back-end}). Zawiera on implementację równania \emph{odwrotnych godzin}(\ref{inhour}):

Na podstawie wyników dopasowano aproksymacje używane  w arkuszu operatorskim to jest:
\begin{align}
	\rho_\$(T) &\approx a_\rho\,T^{b_\rho},\\
	T(\Delta\rho_\$) &\approx a_T\,(\Delta\rho_\$)^{b_T},
\end{align}
co pozwala szybko oszacować oczekiwany okres \(T\) dla zadanego kroku \(\Delta\rho_\$\) 

\begin{figure}[H]
	\centering
	\includegraphics[width=\textwidth]{figures/Exel1.png}
	\caption{Arkusz operatorski (\emph{front-end}): wprowadzanie wyników pomiaru wagi pręta, wyznaczenie \(W_{\mathrm{tot}}\),
		\(\Delta\rho_\$\), szacowanego okresu \(T\) oraz tabeli kroków \((x_k,\Delta h_k)\); wykres charakterystyki pręta z zaznaczonymi krokami.}
	\label{fig:excel_frontend}
\end{figure}

\begin{figure}[H]
	\centering
	\includegraphics[width=\textwidth]{figures/Exel2.png}
	\caption{Arkusz obliczeniowy (\emph{back-end}): implementacja równania inhour \eqref{eq:inhour_dollar_excel} (model 6-grupowy),
		tabela \(\beta_i\), \(\lambda_i\) oraz wartości \(\rho_\$(T)\) dla wybranych okresów; dopasowania potęgowe wykorzystywane w części operatorskiej.}
	\label{fig:excel_backend}
\end{figure}
\label{subsubsec:excel_screeny}








}
\section{Wnioski}


\begin{enumerate}
	\item Na podstawie równań kinetyki punktowej wyprowadzono równanie odwrotnych godzin w zapisie dolarowym oraz pokazano, że po dostatecznie długim czasie odpowiedź reaktora przechodzi w stan asymptotyczny opisany funkcją wykładniczą $n(t)=n_0\exp(t/T)$, co umożliwia praktyczne wyznaczanie reaktywności z okresu reaktora.
	
	\item Opracowano narzędzia obliczeniowe w języku Python do analizy danych z systemu PRETY: dopasowania funkcji $y(t)=a\exp(t/T)$ metodą nieliniowych najmniejszych kwadratów, wyznaczania $\rho_{\$}(T)$ z równania odwrotnych godzin (model 6-grupowy) oraz propagacji niepewności w kolejnych krokach pomiarowych.
	
	\item Porównanie modelu jednogrupowego i sześciogrupowego równania odwrotnych godzin wykazało, że aproksymacja jedno grupowa jest konserwatywna (dla tej samej reaktywności daje krótszy okres reaktora), natomiast model sześciogrupowy jest dokładniejszy, ponieważ uwzględnia 6 grup neutronów opóźnionych, a nie jedną zbiorczą.
	
	\item Pomiary metodą okresu wykonane dla dwóch detektorów (IRL-2.1 oraz IRL-2.2) dały zgodne wartości okresu $T$ oraz odpowiadającej mu reaktywności $\rho_{\$}$ w kolejnych zrzutach, co potwierdza powtarzalność i spójność danych pomiarowych.
	
 	\item Pierwszy krok pomiarowy (zrzut01) charakteryzował się wydłużonym okresem oraz zaniżoną wyznaczoną reaktywnością. Najbardziej prawdopodobne wyjaśnienie jest takie, że PK jest poniżej dna rdzenia reaktora, o czym może świadczyć rys. \ref{zrzut001}, gdzie bardzo dobrze można zauważyć zagłębienie na wykresie, które wskazuje na wprowadzenie ujemnej reaktywności do reaktora. Drugim czynnikiem, który mógł mieć wpływ, to po osiągnięciu dna reaktora pręt kontrolny nie zmienia już swojego położenia, natomiast linka może nadal się odkształcać sprężyście lub być dalej odwijana przez mechanizm, co może skutkować rozjazdem między wskazaniem układu pozycjonowania (PK) a rzeczywistym położeniem pręta. Taki błąd wprowadza stałe przesunięcie osi $z$ w skumulowanej charakterystyce $\rho_{\Sigma}(z)$, a w konsekwencji zafałszowanie reaktywności wyznaczonej dla tego kroku.
	Należałoby zastosować pomiar naprężenia linki tensometrem co pozwoliłoby jednoznacznie wykryć moment osiągnięcia dna przez pręt.

	\item  Ponieważ w pierwszym kroku pomiarowym mogło wystąpić stałe przesunięcie osi położenia wykonano również dopasowanie rozszerzone z parametrem przesunięcia $z_0$ (przyjęto $z_{\mathrm{rel}} = x - z_0$). Uwzględnienie $z_0$ prowadzi do jeszcze mniejszej wartości $H$ (rzędu $1007.5 \pm 19.2\,\mathrm{mm}$) oraz poprawy dopasowania ($R^2 \approx 0.999$). Wynik wymaga weryfikacji w kolejnych pomiarach.
	
	\item Przy porównaniu metody okresu, z metodą zrzutu można doszukać się pewnej nieściśliwości związanej z rozbieżnością wagi tego samego pręta w badaniach, co wymaga dalszych badań. 
	

	

	
	
\end{enumerate}

 

\vfill % umieszcza resztę tekstu na dole strony

\begin{thebibliography}{9}

\bibitem{link} https://www.nuclear-power.com/nuclear-power/reactor-physics/reactor-dynamics/inhour-equation-reactor-kinetics/
\bibitem{Tadi} K. Pytel, {\it Kurs dla operatorów reaktora MARIA}, materiały wewnętrzne NCBJ, 2015
\bibitem{kubowski} J. Kubowski, {\it Efektywneparametry równań kinetyki reaktora badawczego MARIA}, opracowanie wewnętrzne IBJ, O-85/XI/77, 1977
\bibitem{raport5} Eksploatacyjnego Raportu Bezpieczeństwa reaktora MARIA. Rozdział 5 
\bibitem{Lamarsh} Johm R. Lamarsh, Anthony J. Baratta, {\it Introduction to Nuclear Engineering, Third Edition,Rozdział 7.2 REACTOR KINETICS}, Całość cytatu tłumaczona samodzielnie przez autora pracy.
\bibitem{ILR}Z. Marcinkowska, T. Machtyl, M. Marcinkowski{Teoretyczny sygnał z toru ILR w reaktorze MARIA w przypadku urwania pręta kompensacyjnego}, raport wewnetrzny NCBJ B-6/24
\bibitem{OT-12} Krzysztof Pytel, Bolesław Broda {Metody kalibracji ruchomych elementów rdzenia reaktora MARIA}
\end{thebibliography}

\end{document}