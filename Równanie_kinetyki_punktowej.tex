\subsection{Klasyfikacja problemu czasowego}
Procesy zachodzące w reaktorze można podzielić na trzy grupy:
\begin{itemize} 
\item zjawiska długoterminowe trwające dni, lata (wypalenie paliwa, zatrucie bloków berylowych).
\item zjawiska o średniej stałej czasowej które trwają minuty, godziny (zatrucie ksenonem, czy samarem).
\item zjawiska bardzo szybkie, które zachodzą nawet w części sekundy (zrzucenie prętów kontrolnych, załączenie obiegu pierwotnego do chłodzenia reaktora, czy awarie z udziałem prętów pochłaniających, czy układu chłodzenia).
\end{itemize}
 Zakłada się, że reaktor pracuje jako punkt, stąd też nazwa kinetyka punktowa. Takie podejście to podstawa dla większości analiz systemów reaktorowych \cite{Lamarsh}. Równania kinetyki punktowej wyglądają następująco: 
\begin{equation}
	\frac{\mathrm{d}n(t)}{\mathrm{d}t}=\frac{\rho(t)-1)\beta}{\Lambda}n(t)+\sum_{i=1}^{j} \lambda_iC_i(t) 
	\label{nodt}
\end{equation}

\begin{equation}
	\frac{\mathrm{d}C_i(t)}{\mathrm{d}t}=\frac{\beta_i}{\Lambda}n(t)-\lambda_iC_i(t) 
	\label{ciodt}
\end{equation}

\noindent gdzie:

\noindent $n(t)$  jest zmieniającą się w czasie, $t$, gęstością neutronów,

\noindent $\rho_{\$}(t)$  jest reaktywnością wprowadzaną do reaktora, podaną w jednostkach~,\$


\noindent $\beta_i$  jest udziałem neutronów opóźnionych i fotoneutronów z $i$-tej grupy w całej puli neutronów, 

\noindent $\beta$   jest udziałem wszystkich neutronów opóźnionych i fotoneutronów w całej puli neutronów, $\beta~=~\sum\limits_{i=1}^j \beta_i$,

\noindent $\Lambda$  jest czasem życia jednej generacji neutronów,

\noindent $\lambda_i$   jest stałą rozpadu prekursorów neutronów opóźnionych i fotoneutronów z $i$-tej grupy,

\noindent $C_i(t)$ jest koncentracją prekursorów neutronów opóźnionych i fotoneutronów z $i$-tej grupy. 

Równania te możemy zapisać w postaci macierzowej, jako: 


\[
\frac{d}{dt}\;
\underbrace{\begin{bmatrix}
		n(t)\\[2pt]
		C_{1}(t)\\[2pt]
		C_{2}(t)\\[2pt]
		\vdots\\[2pt]
		C_{j}(t)
\end{bmatrix}}_{\vect f(t)}
\;=\;
\underbrace{\begin{bmatrix}
		\dfrac{(\rho_s(t)-1)\,\beta}{\Lambda} & \lambda_1 & \lambda_2 & \cdots & \lambda_j\\[3mm]
		\dfrac{\beta_1}{\Lambda} & -\lambda_1 & 0 & \cdots & 0\\
		\dfrac{\beta_2}{\Lambda} & 0 & -\lambda_2 & \ddots & \vdots\\
		\vdots & \vdots & \ddots & \ddots & 0\\
		\dfrac{\beta_j}{\Lambda} & 0 & \cdots & 0 & -\lambda_j
\end{bmatrix}}_{\mat M(t)}
\;
\underbrace{\begin{bmatrix}
		n(t)\\[2pt]
		C_{1}(t)\\[2pt]
		C_{2}(t)\\[2pt]
		\vdots\\[2pt]
		C_{j}(t)
\end{bmatrix}}_{\vect f(t)}\,.
\]

