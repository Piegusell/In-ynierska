 Podanie ogólnego rozwiązania równania (\ref{maciero}), które opisane jest wzorem (\ref{jednorozw}) sprowadza się do obliczenia wyrażenia $\exp(\mathbb{A}t)$. Należy tu skorzystać z kilku własności algebry macierzy.

\begin{enumerate}[label=\Roman*.]
	\item Jeżeli macierz $\mathbb{B}$ jest diagonalna, to:
	
	\begin{equation}
		\exp(\mathbb{B})=\exp
		\left[\begin{array}{ccc}
			\uu{b_{1,1}}&&\\
			&\ddots&\text{\huge0}\\
			\text{\huge0}&&\uu{b_{i,i}}
		\end{array}\right]=
		\left[\begin{array}{ccc}
			\uu{\exp(b_{1,1})}&&\\
			&\ddots&\text{\huge0}\\
			\text{\huge0}&&\uu{\exp(b_{i,i})}
		\end{array}\right]
		\label{expM}
	\end{equation}
	
	\item Jeżeli macierz jest diagonalna, to wartości na diagonali są również wartościami własnymi tej macierzy.
	
	\item Jeżeli macierz $\mathbb{A}$ jest niediagonalna, ale diagonalizowalna, to można ją zdiagonalizować i otrzymać diagonalną macierz $\mathbb{B}$:
	
	\begin{equation}
		\mathbb{B} = \mathbb{S}^{-1}\mathbb{A} \ \mathbb{S}
		\label{MN}
	\end{equation}
	
	gdzie macierz $\mathbb{S}$ jest macierzą zmiany bazy.
	
	Diagonalizacja macierzy odbywa się poprzez odpowiednie obracanie lub/i skalowanie jej elementów. W przypadku rozwiązywania układu równań należy unikać skalowania, ponieważ zmieniłoby ono sens początkowych równań. Metodą na zdiagonalizowanie macierzy jedynie za pomocą obrotów jest \textbf{diagonalizacja metodą Jordana}. Dowód na to, że rozważana macierz $\mathbb{A}$ jest diagonalizowalna zostanie przedstawiony w rozdziale \ref{jednogrupowy} i \ref{wielogrupowe}.
	
	\item Jeżeli $\mathbb{A} = \mathbb{S}\mathbb{B} \mathbb{S}^{-1}$, to:
	
	\begin{equation}
		\mathbb{A}^n = 
		\underbrace{
			\mathbb{S}\mathbb{B}\cancel{\mathbb{S}}^{-1}\cdot\cancel{\mathbb{S}}\mathbb{B}\cancel{\mathbb{S}}^{-1}\cdot\cdot\cdot \cancel{\mathbb{S}}\mathbb{B}\mathbb{S}^{-1}
		}_{n \ \mathrm{razy}}
		= \mathbb{S}\mathbb{B}^n\mathbb{S}^{-1}
		\label{NN}
	\end{equation}
	
	\item Ponieważ funkcję $exp$ można rozwinąć w szereg Taylora, czyli w szereg wielomianowy: $\exp(x)=\sum\limits_{n=1}^{\infty}\frac{x^n}{n!}$, to korzystając z własności (\ref{NN}) otrzymujemy:
	
	\begin{equation}
		\exp(\mathbb{A}t)=
		\sum\limits_{n=1}^{\infty}\frac{(\mathbb{A}t)^n}{n!}=
		\sum\limits_{n=1}^{\infty}\frac{\mathbb{S}(\mathbb{B}t)^n\mathbb{S}^{-1}}{n!}=
		\mathbb{S}\left(\sum\limits_{n=1}^{\infty}\frac{(\mathbb{B}t)^n}{n!}\right)\mathbb{S}^{-1}=
		\mathbb{S} \exp(\mathbb{B}t)\mathbb{S}^{-1}
		\label{wielo}
	\end{equation}
	
\end{enumerate}

\noindent Korzystając z wymienionych powyżej własności I--V, układ równań (\ref{maciero}) można rozwiązać w~następujący sposób:

\begin{enumerate}
	
	\item Zapisać układ równań kinetyki w postaci macierzowej. Upewnić się, że macierz współczynników (macierz $\mathbb{A}$) jest niezależna od czasu.
	
	\item Zdiagonalizować macierz współczynników (macierz $\mathbb{A}$) metodą Jordana i otrzymać diagonalną macierz $\mathbb{B}$.
	
	\item Otrzymać rozwiązanie ogólne układu równań, którym będzie:
\end{enumerate}
	\begin{equation}
		\vec{f}(t)=\mathbb{S}\exp(\mathbb{B}t)\mathbb{S}^{-1}\vec{f}(0)
		\label{rozwogolne}
	\end{equation}
	\textbf{Jeżeli macierz jest kwadratowa o wymiarze $n\times n$ i posiada $n$ różnych wartości własnych, to jest diagonalizowalna.}
	Poniżej wykażemy, że analizowana macierz kwadratowa $\mathbb{A}$ o wymiarze \mbox{$(j+1)\times (j+1)$} spełnia powyższy warunek.
	
	Dla uproszczenia zapisu rozważaną macierz $\mathbb{A}$ przedstawimy za pomocą elementów macierzy oznaczonych małymi literkami $a$ z odpowiednimi indeksami w następujący sposób:
	\begin{equation}
		\mathbb{A}= 
		\left[\begin{array}{ccccc}
			\frac{\rho_0-\beta}{\Lambda}&\lambda_1&\lambda_2&\hdots&\lambda_j\\
			\uu{\frac{\beta_1}{\Lambda}}&-\lambda_1&&\text{\huge0}&\\
			\uu{\frac{\beta_2}{\Lambda}}&&-\lambda_2&&\\
			\uu{\vdots}&\text{\huge0}&&\ddots&\\
			\uu{\frac{\beta_j}{\Lambda}}&&&&-\lambda_j
		\end{array}\right]
		=
		\left[\begin{array}{ccccc}
			\uu{a_{1,1}}&a_{1,2}&a_{1,3}&\hdots&a_{1,j+1}\\
			\uu{a_{2,1}}&a_{2,2}&&\text{\huge0}&\\
			\uu{a_{3,1}}&&a_{3,3}&&\\
			\uu{\vdots}&\text{\huge0}&&\ddots&\\
			\uu{a_{j+1,1}}&&&&a_{j+1,j+1}
		\end{array}\right]
		\label{macierouuu}
	\end{equation}
	
	%\noindent Ze względu na kształt macierzy $\mathbb{A}$, wiadomo, że poszczególne wartości $\lambda_i$ będą różne, ponieważ reprezentują stałe rozpadu, charakteryzujące różne grupy neutronów opóźnionych i fotoneutronów. Różna od nich będzie także wartość $\frac{\rho_0-\beta}{\Lambda}$. Oznacza to, że macierz $\mathbb{A}$ będzie posiadała $j+1$ różnych od siebie, wyrazów $a_{i,i}$ na diagonali.
	%\noindent Poniżej dowiedziemy, że macierz  $\mathbb{A}$ posiada $(j+1)$ różnych wartości własnych. 
	
	\noindent Jeżeli wektor $\vec{v}_k$ jest \textbf{wektorem własnym} macierzy $\mathbb{A}$, to zgodnie z jego definicją jest \textbf{niezerowy} i spełnia następujące równanie: 
	
	\begin{equation}
		\mathbb{A}\cdot \vec{v}_k = \alpha_k \cdot \vec{v}_k 
		\label{czczczczcz}
	\end{equation}
	
	\noindent gdzie $\alpha_k$ jest wartością własną, stowarzyszoną z tym wektorem własnym.
	
	Jeżeli zdefiniujemy macierz $\mathbb{M}(\alpha_k)$, jako:
	
	\begin{equation}
		\mathbb{M}(\alpha_k)= 
		\mathbb{A}-\alpha_k\cdot \mathbb{1}
		\label{yuyuyuyuyuy}
	\end{equation}
	
	\noindent gdzie macierz $\mathbb{1}$ jest macierzą jednostkową, to:
	
	\begin{equation}
		\mathbb{M}(\alpha_k)\cdot \vec{v}_k= 0
		\label{brbrbrbrb}
	\end{equation}
	
	\noindent ponieważ:
	
	\begin{equation}
		(\mathbb{A}-\alpha_k\cdot \mathbb{1})\cdot \vec{v}_k=\mathbb{A}\cdot \vec{v}_k -\alpha_k \cdot \vec{v}_k =  0
		\label{ioioioioi}
	\end{equation}
	
	\noindent Gdyby macierz $\mathbb{M}$ była macierzą odwracalną, to istniałaby macierz do niej odwrotna, czyli $\mathbb{M}^{-1}$. Wtedy, mnożąc równanie  (\ref{brbrbrbrb}) obustronnie przez macierz $\mathbb{M}^{-1}$ otrzymalibyśmy:
	
	\begin{equation}
		\underbrace{
			\mathbb{M}^{-1}\cdot \mathbb{M}
		}_{\ \Resize{0.25cm}{\mathbb{1}}}
		\cdot \vec{v}_k =
		\underbrace{
			\mathbb{M}^{-1} \cdot 0
		}_{\ \Resize{0.2cm}{0}}
		\label{udtwwjdu}
	\end{equation}
	
	\noindent a powyższe równanie (\ref{udtwwjdu}) byłoby spełnione jedynie wtedy, gdyby wektor $\vec{v}_k$ był zerowy, czyli $\vec{v}_k =0$. A to jest sprzeczne z definicją wektora własnego, który musi być niezerowy. Oznacza to, że macierz $\mathbb{M}$ \textbf{nie jest macierzą odwracalną}. A \textbf{wyznacznik każdej macierzy nieodwracalnej jest równy zero}. Tak więc:
	
	\begin{equation}
		\det \mathbb{M}(\alpha_k)=
		\det
		(\mathbb{A}-\alpha_k \cdot \mathbb{1}) 
		= 0
		\label{ertyerty}
	\end{equation}
	
	\noindent Wobec powyższego wartości własne macierzy $\mathbb{A}$, oznaczone jako $\alpha_k$, spełniają równanie (\ref{ertyerty}).
	
	
	Definiujemy macierz $\mathbb{M}(\alpha)$, jako funkcję ciągłego parametru $\alpha$:
	
	\begin{equation}
		\mathbb{M}(\alpha)= 
		\mathbb{A}-\alpha\cdot \mathbb{1}
		=
		\left[\begin{array}{ccccc}
			\uu{a_{1,1}-\alpha}&a_{1,2}&a_{1,3}&\hdots&a_{1,j+1}\\
			\uu{a_{2,1}}&a_{2,2}-\alpha&&\text{\huge0}&\\
			\uu{a_{3,1}}&&a_{3,3}-\alpha&&\\
			\uu{\vdots}&\text{\huge0}&&\ddots&\\
			\uu{a_{j+1,1}}&&&&a_{j+1,j+1}-\alpha
		\end{array}\right]
		\label{mmmmmmmmm}
	\end{equation}
	Wyznacznik tej macierzy, $\det(\mathbb{M}(\alpha))$, będzie funkcją wielomianową parametru $\alpha$. Zgodnie z~równaniem (\ref{ertyerty}), wartości własne $\alpha_k$ będą miejscami zerowymi (pierwiastkami) tego wielomianu, czyli:
	
	\begin{equation}
		\det
		(\mathbb{M}(\alpha_k)) 
		= 0
		\label{wewewewewewe}
	\end{equation}
	
	\subsection{Przypadek jednogrupowy, $j=1$} \label{jednogrupowy}
	
	W przypadku jednej grupy neutronów ($j=1$) wzór (\ref{wewewewewewe}) przybierze postać:
	\begin{equation}
		\det(\mathbb{M}(\alpha)) =
		\left|\begin{array}{cc}
			\frac{\rho_0-\beta}{\Lambda} - \alpha&\lambda_1\\
			\uu{\frac{\beta}{\Lambda}}&-\lambda_1 - \alpha
		\end{array}\right|
		= (\frac{\rho_0-\beta}{\Lambda} - \alpha)(-\lambda_1 - \alpha) - \frac{\beta}{\Lambda}\lambda_1
		\label{babel}
	\end{equation}
	
	\noindent Podstawiając równanie (\ref{babel}) do równania (\ref{wewewewewewe}) i rozwiązując je (czyli znajdując jego miejsca zerowe), otrzymamy dwie różne wartości własne: 
	
	\begin{equation}
		\alpha_{\pm}=-\frac{\Lambda\lambda-\rho_0+\beta}{2\Lambda}\pm\frac{\sqrt{\Lambda^2\lambda^2+2\Lambda\lambda(\rho_0+\beta)+\rho_0^2+\beta^2-2\rho_0\beta}}{2\Lambda}
		\label{apm}
	\end{equation}
	
	\noindent Ponieważ w tym przypadku (jednogrupowym) macierz $\mathbb{A}$ o wymiarze $2\times 2$ posiada dwie różne wartości własne, co oznacza, że \textbf{jest macierzą diagonalizowalną}.
	
	W przypadku jednogrupowym, który jest stosunkowo prosty do przeprowadzenia dokładnego, analitycznego rozwiązania, macierz diagonalna $\mathbb{B}$ będzie równa:
	
	\begin{equation}
		\mathbb{B}=\left[\begin{array}{cc}
			\alpha_+&0\\
			\uu{0}&\alpha_-
		\end{array}\right]
		\label{bee}
	\end{equation}
	
	\noindent Następnie należy znaleźć macierz obrotu $\mathbb{S}$. Będzie złożona z wektorów własnych $\vec{\mathfrak{s}}$ macierzy $\mathbb{A}$, spełniających równanie:
	
	\begin{equation}
		\mathbb{A}\vec{\mathfrak{s}}_{\pm}=\alpha_{\pm}\vec{\mathfrak{s}}_{\pm}
		\label{vecss}
	\end{equation}
	
	\noindent Wtedy macierz obrotu S będzie równa:
	
	\begin{equation}
		\mathbb{S}=[\vec{\mathfrak{s}}_+,\vec{\mathfrak{s}}_-]=
		\left[\begin{array}{cc}
			\frac{\Lambda}{\beta}(\lambda+\alpha_+)&\frac{\Lambda}{\beta}(\lambda+\alpha_-)\\
			\uu{1}&1
		\end{array}\right]=
		\left[\begin{array}{cc}
			x_+&x_-\\
			\uu{1}&1
		\end{array}\right]
		\label{svec}
	\end{equation}
	
	\noindent gdzie dla uproszczenia zapisu przyjęto, że:
	
	\begin{equation}
		x_{\pm}=\frac{\Lambda}{\beta}(\lambda+\alpha_{\pm})
		\label{ixsde}
	\end{equation}
	
	\noindent W kolejnym kroku należy obliczyć macierz odwrotną $\mathbb{S}^{-1}$. W tym przypadku będzie ona równa:
	
	\begin{equation}
		\mathbb{S}^{-1}=\frac{1}{\det[\mathbb{S}]}
		\left[\begin{array}{cc}
			\uu{1}&-\frac{\Lambda}{\beta}(\lambda+\alpha_-)\\
			\uu{-1}&\frac{\Lambda}{\beta}(\lambda+\alpha_+)
		\end{array}\right]=\frac{1}{x_+-x_-}
		\left[\begin{array}{cc}
			\uu{1}&-x_-\\
			\uu{-1}&x_+
		\end{array}\right]
		\label{s-1}
	\end{equation}
	
	\noindent Zgodnie z (\ref{expM}) otrzymujemy, że:
	
	\begin{equation}
		\exp(\mathbb{B}t)=
		\left[\begin{array}{cc}
			\exp(\alpha_+t)&0\\
			\uu{0}&exp(\alpha_-t)
		\end{array}\right]=
		\left[\begin{array}{cc}
			\mathfrak{e}_+&0\\
			\uu{0}&\mathfrak{e}_-
		\end{array}\right]
		\label{expBe}
	\end{equation}
	
	\noindent gdzie dla uproszczenia zapisu przyjęto, że:
	
	\begin{equation}
		\mathfrak{e}_{\pm}=\exp(\alpha_{\pm}t)
		\label{eee}
	\end{equation}
	
	\noindent Aby otrzymać rozwiązanie ogólne rozważanego jednogrupowego przypadku, należy skorzystać ze wzoru (\ref{rozwogolne}):
	
	\begin{equation}
		\begin{aligned}
			\left[\begin{array}{c}
				n(t)\\
				\uu{C(t)}
			\end{array}\right]=&\frac{1}{x_+-x_-}
			\left[\begin{array}{cc}
				x_+&x_-\\
				\uu{1}&1
			\end{array}\right]
			\left[\begin{array}{cc}
				\mathfrak{e}_+&0\\
				\uu{0}&\mathfrak{e}_-
			\end{array}\right]
			\left[\begin{array}{cc}
				1&-x_-\\
				\uu{-1}&x_+
			\end{array}\right]
			\left[\begin{array}{c}
				n_0\\
				\uu{C_0}
			\end{array}\right]=\\
			&\\
			&=\frac{1}{x_+-x_-}
			\left[\begin{array}{cc}
				x_+\mathfrak{e}_+-x_-\mathfrak{e}_-&x_+x_-(\mathfrak{e}_--\mathfrak{e}_+)\\
				\uu{\mathfrak{e}_+-\mathfrak{e}_-}&x_+\mathfrak{e}_--x_-\mathfrak{e}_+
			\end{array}\right]
			\left[\begin{array}{c}
				n_0\\
				\uu{C_0}
			\end{array}\right]
			\label{jednorozwog}
		\end{aligned}
	\end{equation}
	
	\noindent Jak widać z (\ref{jednorozwog})rozwiązanie ogólne prostego, jednogrupowego przypadku jest skomplikowane. Ale obserwując jego kształt, można zauważyć, że w tym przypadku, gdy mamy dwa równania kinetyki (tzn. wymiar macierzy $\mathbb{A}$ równy jest 2), to w rozwiązaniach ogólnych wystąpią wyrażenia liniowe dwóch różnych eksponentów. Wykorzystując fakt, że zarówno iloczyn, jak i iloraz eksponentów również jest eksponentem, liniowe wyrażenia eksponentów można przekształcić do odpowiedniej sumy eksponentów o odpowiednich wykładnikach. To oznacza, że wzór (\ref{jednorozwog})można uprościć do postaci:
	
	\begin{equation}
		n(t)=N_+e^{\alpha_+t}+N_-e^{\alpha_-t}
		\label{n_suma_exp}
	\end{equation}
	
	\begin{equation}
		C(t)=C_{+}e^{\alpha_+t}+C_{-}e^{\alpha_-t}
		\label{C_suma_exp}
	\end{equation}
	
	\noindent gdzie:
	
	\begin{equation}
		N_{\pm}=\pm \frac{n_0x_{\pm}-C_0x_+x_-}{x_+-x_-}
		\label{nenenenenenenenne}
	\end{equation}
	
	\noindent oraz 
	
	\begin{equation}
		C_{\pm}=\pm \frac{n_0-C_0x_{\mp}}{x_+-x_-}
		\label{yuertueyt}
	\end{equation}
\newpage
\subsection{Rozwiązanie wielogrupowe, $j=6$}\label{wielogrupowe}
W sytuacji kiedy chcemy rozdzielić neutrony opóźnione na sześć różnych grup, musimy  rozwiązać  układ kinetyki składający się aż z  siedmiu równań, które przy rozwiązywaniu analitycznym obarcza nas niepotrzebnym chaosem. 
Dlatego też dla przypadku sześciogrupowego  $\mathbb{M}$ skorzystamy z obliczeń numerycznych, w którym niepodzielnie rządzi porządek i precyzja. 


Podział grupowy neutronów opóźnionych wraz ze stałymi zaniku ich prekursorów oraz odpowiadającymi im czasami połowicznego zaniku i udziałami poszczególnych grup w~bilansie neutronów reaktora MARIA \cite{kubowski}, przedstawiono w~Tabeli~\ref{tab2}. Wartości te zostały użyte do obliczenia wyznacznika macierzy $\mathbb{M}$. Na Rys. \ref{iretwor} przedstawiono obliczony numerycznie wyznacznik macierzy $\mathbb{M}$, jako funkcję parametru $\alpha$ dla trzech wartości wprowadzonej do rdzenia reaktywności, wynoszących, odpowiednio, 0,5 \$, 1 \$ oraz 1,5 \$. Miejsca zerowe oznaczono punktami, a ich wartości przedstawiono w Tabeli \ref{tab1}. Jako czas życia jednej generacji neutronów, przyjęto $\Lambda = 146 \ \mu \mathrm{s}$.

 
\begin{table}[H]
	\centering
	\caption{Udziały, stałe zaniku oraz czasy połowicznego zaniku prekursorów neutronów opóźnionych \cite{kubowski}.}
	\label{tab2}
	\renewcommand{\arraystretch}{1.2}
	\begin{tabular}{|c|c|c|c|c|}
		\hline
		Przypadek & Nr grupy $i$ & $\beta_i$ & $\lambda_i$, $\mathrm{s^{-1}}$ & $T_{1/2}$, s \\
		\hline
		\multirow{6}{*}{Wielogrupowy}
		& 1 & $2{,}43 \cdot 10^{-4}$ & $1{,}27 \cdot 10^{-2}$ & 54{,}6 \\
		& 2 & $1{,}363 \cdot 10^{-3}$ & $3{,}17 \cdot 10^{-2}$ & 21{,}9 \\
		& 3 & $1{,}203 \cdot 10^{-3}$ & $1{,}15 \cdot 10^{-1}$ & 6{,}03 \\
		& 4 & $2{,}605 \cdot 10^{-3}$ & $3{,}11 \cdot 10^{-1}$ & 2{,}20 \\
		& 5 & $8{,}19 \cdot 10^{-4}$ & $1{,}40$ & 0{,}50 \\
		& 6 & $1{,}67 \cdot 10^{-4}$ & $3{,}87$ & 0{,}18 \\
		\hline
		Jednogrupowy
		& \multicolumn{2}{c|}{$\beta = \sum\limits_{i=1}^{6} \beta_i = 6{,}4 \cdot 10^{-3}$}
		& \multicolumn{2}{c|}{$\hat{\lambda} = 0{,}44~\mathrm{s^{-1}}$} \\
		\hline
	\end{tabular}
\end{table}



\begin{table}[H]
	\centering
	\caption{Miejsca zerowe wyznacznika $\det\mathbb{M}(\alpha_i)$ dla wybranych wartości wprowadzonej reaktywności.}
	\label{tab1}
	\renewcommand{\arraystretch}{1.2}
	\begin{tabular}{|c|c|c|c|}
		\hline
		\multirow{2}{*}{$\alpha_i$} &
		\multicolumn{3}{c|}{Reaktywność $\rho$} \\ \cline{2-4}
		& $\rho = 0{,}5~\$$ & $\rho = 1{,}0~\$$ & $\rho = 1{,}5~\$$ \\
	\hline
	$\alpha_1$ & 0,180  & 3,729  & 22,707 \\
	$\alpha_2$ & -0,013 & -0,013 & -0,013 \\
	$\alpha_3$ & -0,047 & -0,039 & -0,037 \\
	$\alpha_4$ & -0,161 & -0,140 & -0,132 \\
	$\alpha_5$ & -1,113 & -0,756 & -0,500 \\
	$\alpha_6$ & -3,683 & -2,936 & -1,770 \\
	$\alpha_7$ & -22,820& -5,634 & -4,078 \\
	\hline
	
	\end{tabular}
\end{table}

\begin{figure}[H]
	\centering
	\includegraphics[width= 16cm]{figures/M(alfa)_rho_all.png}
	\caption{Obliczona wartość wyznacznika macierzy $\mathbb{M}$, jako funkcja parametru $\alpha$ dla trzech wartości wprowadzonej do rdzenia reaktywności, wynoszących, odpowiednio, 0,5 \$, 1 \$ oraz 1,5 \$. Miejsca zerowe wyznacznika zaznaczono na rysunku za pomocą punktów. Poszczególne kolory odpowiadają wartościom wprowadzonej reaktywności w \$.}\label{iretwor}
\end{figure}

Jak widać wyznacznik macierzy $\mathbb{M}$ dla przypadku sześciogrupowego posiada siedem różnych miejsc zerowych, będących jednocześnie siedmioma różnymi wartościami własnymi macierzy $\mathbb{A}$, więc macierz $\mathbb{A}$ \textbf{jest macierzą diagonalizowalną}. I analogicznie do przypadku jednogrupowego, rozwiązania równań kinetyki punktowej (\ref{maciero})będą liniowymi kombinacjami siedmiu eksponentów, które można przekształcić do postaci sumy siedmiu eksponentów z odpowiednimi współczynnikami:

\begin{equation}
	n(t)=\sum_{k=1}^7 N_ke^{\alpha_kt}
	\label{werty}
\end{equation}

\begin{equation}
	C_i(t)=\sum_{k=1}^7 C_{k,i}e^{\alpha_kt}
	\label{dfghj}
\end{equation}

\noindent ze względu na skomplikowaną postać rozwiązań równań kinetyki punktowej w przypadku sześciu grup neutronów opóźnionych analityczne postaci współczynników $N_k$ oraz $C_{k,i}$ nie zostaną podane. 

Warto także podkreślić, że wśród otrzymanych wartości własnych, zawsze tylko jedna wartość własna jest dodatnia, a pozostałe są ujemne. Oznacza to, że dla przypadku każdej z trzech zastosowanych reaktywności tylko eksponent z dodatnią wartością własną będzie "wybuchać" w nieskończoności, natomiast wszystkie pozostałe (z wartościami ujemnymi) zanikną.
	