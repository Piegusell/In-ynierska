W celu  łatwiejszego zobrazowania, posłużono się diagramem \ref{diagram}, który przedstawia przebieg i zależność treści występujących w pracy dyplomowej.






\begin{diagram}[H]
	\centering
	\caption{przedstawiający przebieg pracy dyplomowej w sposób łatwy do przyswojenia.}
	\label{diagram}


	\begin{adjustbox}{max totalsize={\textwidth}{0.92\textheight},center}
		\begin{tikzpicture}[
			font=\small,
			node distance=10mm and 18mm,
			>=Latex,
			block/.style={
				draw, rounded corners=3mm, align=center,
				minimum width=62mm, minimum height=16mm, inner sep=4mm
			},
			blockWide/.style={
				draw, rounded corners=4mm, align=center,
				minimum width=150mm, minimum height=18mm, inner sep=5mm
			},
			blockBig/.style={
				draw, rounded corners=4mm, align=center,
				minimum width=70mm, minimum height=26mm, inner sep=5mm
			},
			blockExel/.style={
				draw, rounded corners=4mm, align=center,
				minimum width=40mm, minimum height=26mm, inner sep=5mm
			},
			smallBlock/.style={
				draw, rounded corners=3mm, align=center,
				minimum width=62mm, minimum height=12mm, inner sep=3.5mm
			}
			]
			
			% ====== GÓRA ======
			\node[blockWide] (cel) {%
				\textbf{CEL:} Charakterystyka reaktywnościowa pręta pochłaniającego\\
				-- wyznaczenie teoretycznie i porównanie z pomiarem.
			};
			
			
			
			% ====== START: Dwie kolumny ======
			\node[block, below=8mm of cel.south west, anchor=north west] (budowa) {%
				Budowa reaktora\\
				funkcja prętów pochłaniających
			};
			
			\node[blockExel, below=8mm of cel.south east, anchor=north east] (kinetyka) {%
				Równanie\\
				kinetyki\\
				punktowej
			};
			
			% ====== LEWA GAŁĄŹ 
			\node[block, below=9mm of budowa] (efektywnosc) {
				Teoretyczna efektywność\\
				pręta pochłaniającego\\
				na podstawie\\
				rozkładu strumienia neutronów
			};
			
			\node[block, below=9mm of efektywnosc] (krzywaS) {%
				Teoretyczna\\
				krzywa „S”
			};
			\node[blockBig, below=16mm of krzywaS] (dofitowanie) {%
				Dopasowanie\\
				do punktów\\
				pomiarowych\\
				krzywej teoretycznej "S”
			};
			
			\node[blockBig, below=10mm of dofitowanie] (koniec) {%
				
				\textbf{Koniec}
			};
			% ====== PRAWA GAŁĄŹ (metoda pomiaru) ======
			\node[blockExel, below=9mm of kinetyka] (odwrotne) {%
				Równanie\\
				odwrotnych\\
				godzin
			};
			
			\node[blockExel, below=9mm of odwrotne] (metodaOkres) {%
				Metoda\\
				pomiaru reakt.\\
				okresem
			};
			
			
			
			% prawa kontynuacja
			\node[blockExel, below=16mm of metodaOkres] (pomiary) {%
				Pomiary:\\[-1mm]
				
				$\rightarrow$met. zrzutu\\
				$\rightarrow$met. okresu\\
				$\rightarrow$Excel\\
				$\rightarrow$zebrane dane
			};
			
			\node[blockBig, below=10mm of pomiary] (analizaPom) {%
				Analiza pomiarów\\[1mm]
				$\rightarrow$ położenia pręta w kolejnych krokach pomiarowych\\
				$\rightarrow$ okres reaktora w kolejnych krokach pomiarowych\\
				\hspace*{6mm}$\rightarrow$ reaktywność w kolejnych krokach pomiarowych
			};
			
			\node[smallBlock, below=10mm of analizaPom] (punkty) {%
				Punkty pomiarowe\\
				do krzywej „S”
			};
			
			% ====== STRZAŁKI (bez sprzężeń) ======
			\begin{scope}[
				->,
				>=Latex,
				line width=0.6pt,
				draw=black!70
				]
				
				% Rozgałęzienie od CEL do obu kolumn (ładne kąty proste)
				\draw (cel.south) ++(0,-2mm) -| (budowa.north);
				\draw (cel.south) ++(0,-2mm) -| (kinetyka.north);
				
				% Lewa kolumna
				\draw (budowa) -- (efektywnosc);
				\draw (efektywnosc) -- (krzywaS);
				\draw (krzywaS) -- (dofitowanie);
				\draw (dofitowanie) -- (koniec);
				
				% Prawa kolumna
				\draw (kinetyka) -- (odwrotne);
				\draw (odwrotne) -- (metodaOkres);
				\draw (metodaOkres) -- (pomiary);
				\draw (pomiary) -- (analizaPom);
				\draw (analizaPom) -- (punkty);
				\coordinate (kolano) at ($(dofitowanie.east)+(11mm,0)$);
				\coordinate (kolano2) at ($(punkty.west)+(-21.8mm,0)$);
				\draw[->, line width=0.6pt]
				(punkty) -- (kolano2)-- (kolano) -- (dofitowanie.east);
			\end{scope}
			
			
		\end{tikzpicture}
	\end{adjustbox}

\end{diagram}

