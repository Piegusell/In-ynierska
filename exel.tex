

\subsubsection{Cel}
Celem procedury jest wyznaczenie kolejnych przemieszczeń pręta regulacyjnego (w mm) o zbliżonej wartości przyrostom reaktywności wyrażonej w dolarach.




\subsubsection{Zrzuty ekranu arkusza Excel}
Na rys.~\ref{fig:excel_frontend} pokazano arkusz operatorski (warstwa \emph{front-end}), w którym użytkownik
wprowadza wyniki pomiaru wagi pręta (w~\(\$\)) oraz liczbę kroków \(N\). Arkusz automatycznie wyznacza:
\(\,W_{\mathrm{tot}}\) (średnia z trzech pomiarów wagi pręta metodą zrzutu), skok reaktywności dla każdego z planowanych kroków pomiarowych \(\Delta\rho_\$=W_{\mathrm{tot}}/N\), szacowany okres reaktora \(T\)
oraz tabelę kolejnych zagłębień pręta dla projektowanych kroków pomiarowych: skumulowane zagłębienie \(x_k\) i~przyrost w kroku \(\Delta h_k\).
Dodatkowo przedstawiona jest krzywa charakterystyki pręta wraz z naniesionymi punktami, odpowiadającymi krokom, gdzie podczas metody zrzutu[\ref{metoda_zrzutu}] ustalono, że waga PK6 wynosi 1.72\$ co warto mieć w pamięci w podrozdziale \ref{tabele pomiarowe}.
Natomiast na  rys.~\ref{fig:excel_backend} przedstawiono arkusz obliczeniowy (warstwa \emph{back-end}). Zawiera on implementację równania \emph{odwrotnych godzin}(\ref{inhour}):

Na podstawie wyników dopasowano aproksymacje używane  w arkuszu operatorskim to jest:
\begin{align}
	\rho_\$(T) &\approx a_\rho\,T^{b_\rho},\\
	T(\Delta\rho_\$) &\approx a_T\,(\Delta\rho_\$)^{b_T},
\end{align}
co pozwala szybko oszacować oczekiwany okres \(T\) dla zadanego kroku \(\Delta\rho_\$\) 

\begin{figure}[H]
	\centering
	\includegraphics[width=\textwidth]{figures/Exel1.png}
	\caption{Arkusz operatorski (\emph{front-end}): wprowadzanie wyników pomiaru wagi pręta, wyznaczenie \(W_{\mathrm{tot}}\),
		\(\Delta\rho_\$\), szacowanego okresu \(T\) oraz tabeli kroków \((x_k,\Delta h_k)\); wykres charakterystyki pręta z zaznaczonymi krokami.}
	\label{fig:excel_frontend}
\end{figure}

\begin{figure}[H]
	\centering
	\includegraphics[width=\textwidth]{figures/Exel2.png}
	\caption{Arkusz obliczeniowy (\emph{back-end}): implementacja równania inhour \eqref{eq:inhour_dollar_excel} (model 6-grupowy),
		tabela \(\beta_i\), \(\lambda_i\) oraz wartości \(\rho_\$(T)\) dla wybranych okresów; dopasowania potęgowe wykorzystywane w części operatorskiej.}
	\label{fig:excel_backend}
\end{figure}
\label{subsubsec:excel_screeny}



