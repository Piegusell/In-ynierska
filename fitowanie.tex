Podczas przeprowadzania pomiarów metodą zrzutu dla pierwszego kroku pomiarowego można zaobserwować aberrację
w postaci dłuższego niż pozostałych okresów reaktora oraz małej wartości reaktywności
$\rho_\$$ dla pierwszego pomiaru.
Zjawisko to wynika z faktu, że w chwili wykonywania pierwszego wyciągania pręt
regulacyjny PK6 wychodził z rdzenia na jego dole o dodatkowe kilka (nie wiemy ile) milimetrów, oraz linka która jest przyczepiona do PK mogła być rozwijana dalej po osiągnięciu przez pręt dna reaktora.  W konsekwencji wysuwanie pręta prowadziło do
zwiększonego pochłaniania neutronów, co znalazło odzwierciedlenie zarówno
w wydłużeniu okresu reaktora, jak i w obniżonej wartości reaktywności.
Efekt ten jest wyraźnie widoczny na wykresie przedstawionym na
rys.~\ref{zrzut001}.

W rezultacie pierwszy pomiar charakteryzuje się wartościami odbiegającymi
od pozostałych wyników przez co zaniża o nieznaną wartość wszystkie pozostałe punkty.
Pozostałe dziesięć pomiarów wykonano dla położeń pręta bez tego efektu, odpowiadających kolejnym schodkom zanurzenia. Uzyskane dla nich wartości okresu reaktora
oraz reaktywności są do siebie zbliżone i mogą zostać uznane za miarodajne.
\begin{table}[H]
	\centering
	\caption{Wartości okresu reaktora oraz reaktywności wyznaczone na podstawie
		pomiarów detektora IRL-2.1 metodą okresu reaktora.}
	\renewcommand{\arraystretch}{1.2}
	\begin{tabular}{|c|c|c|}
		\hline
		\textbf{Zrzut} &
		$\boldsymbol{T_{\text{IRL-2.1}}}$ [s] &
		$\boldsymbol{\rho_{\text{IRL-2.1}}}$ [\$] \\
	\hline
	zrzut01 & $211,67(5)$  & $0,05200(10)$ \\
	zrzut02 & $76,47(11)$  & $0,11864(13)$ \\
	zrzut03 & $61,73(8)$   & $0,13861(13)$ \\
	zrzut04 & $76,96(14)$  & $0,11808(15)$ \\
	zrzut05 & $63,53(13)$  & $0,1358(2)$   \\
	zrzut06 & $63,45(11)$  & $0,13593(17)$ \\
	zrzut07 & $71,40(12)$  & $0,12480(15)$ \\
	zrzut08 & $76,70(11)$  & $0,11839(13)$ \\
	zrzut09 & $71,69(11)$  & $0,12444(13)$ \\
	zrzut10 & $67,62(14)$  & $0,12985(19)$ \\
	zrzut11 & $67,90(10)$  & $0,12946(13)$ \\
	\hline
	
	\end{tabular}
	\label{tab:IRL21}
\end{table}


\begin{table}[H]
	\centering
	\caption{Wartości okresu reaktora oraz reaktywności wyznaczone na podstawie
		pomiarów detektora IRL-2.2 metodą okresu reaktora.}
	\renewcommand{\arraystretch}{1.2}
	\begin{tabular}{|c|c|c|}
		\hline
		\textbf{Zrzut} &
		$\boldsymbol{T_{\text{IRL-2.2}}}$ [s] &
		$\boldsymbol{\rho_{\text{IRL-2.2}}}$ [\$] \\
\hline
zrzut01 & $212,18(8)$  & $0,05189(17)$ \\
zrzut02 & $76,60(12)$  & $0,11850(14)$ \\
zrzut03 & $61,95(11)$  & $0,13826(17)$ \\
zrzut04 & $76,47(14)$  & $0,11865(16)$ \\
zrzut05 & $63,60(15)$  & $0,1357(2)$   \\
zrzut06 & $63,30(13)$  & $0,13615(20)$ \\
zrzut07 & $71,93(15)$  & $0,12412(19)$ \\
zrzut08 & $76,52(12)$  & $0,11859(14)$ \\
zrzut09 & $71,66(13)$  & $0,12447(16)$ \\
zrzut10 & $67,60(16)$  & $0,12988(22)$ \\
zrzut11 & $67,91(16)$  & $0,12944(22)$ \\
\hline

	\end{tabular}
	\label{tab:IRL22}
\end{table}
Na podstawie średnich wartości reaktywności $\bar{\rho}$ zestawionych w tabeli
\ref{zrzuty} wyznaczono skumulowaną charakterystykę pręta regulacyjnego PK6
w postaci funkcji $\rho_{\Sigma}(s)$, zdefiniowanej jako suma kolejnych
przyrostów reaktywności:
\begin{equation}
	\rho_{\Sigma,k} = \sum_{j=1}^{k} \bar{\rho}_j .
\end{equation}
Niepewność wielkości $\rho_{\Sigma}$ wyznaczono metodą propagacji niepewności
\cite{propagacja}, rozdział~5.1, wzór~(10), przyjmując niezależność niepewności
poszczególnych kroków pomiarowych.


\begin{table}[h]
	\centering
	\caption{Średnie arytmetyczne okresu reaktora oraz reaktywności wyznaczone
		na podstawie pomiarów detektorów IRL-2.1 i IRL-2.2 z położeniem PK6.
		Reaktywność skumulowaną $\rho_{\Sigma}$ oraz jej niepewność wyznaczono
		na podstawie sumowania kolejnych przyrostów reaktywności
		z uwzględnieniem propagacji niepewności.}
	\label{zrzuty}
	\renewcommand{\arraystretch}{1.2}
	\begin{tabular}{|c|c|c|c|c|}
		\hline
		\textbf{Zrzut} &
		$\boldsymbol{\bar{T}}$ s &		$\boldsymbol{\bar{\rho}}$ \$ $  &
		$\boldsymbol{\rho_{\Sigma}} \$ $  &
		$\boldsymbol{z}$ mm \\
		\hline
		zrzut01 & $211{,}92(5)$  & $0{,}05194(1)$ & $0{,}05194(1)$ & $219{,}00(1)$ \\
		zrzut02 & $76{,}54(8)$   & $0{,}11857(9)$ & $0{,}17051(9)$ & $307{,}30(2)$ \\
		zrzut03 & $61{,}84(7)$   & $0{,}13843(11)$& $0{,}30894(14)$& $388{,}90(2)$ \\
		zrzut04 & $76{,}72(9)$   & $0{,}11836(11)$& $0{,}42730(18)$& $450{,}00(1)$ \\
		zrzut05 & $63{,}56(10)$  & $0{,}13575(15)$& $0{,}56305(23)$& $517{,}60(2)$ \\
		zrzut06 & $63{,}37(8)$   & $0{,}13604(13)$& $0{,}69909(27)$& $584{,}00(1)$ \\
		zrzut07 & $71{,}67(9)$   & $0{,}12446(12)$& $0{,}82355(29)$& $646{,}00(1)$ \\
		zrzut08 & $76{,}61(8)$   & $0{,}11849(9)$ & $0{,}94204(31)$& $712{,}00(1)$ \\
		zrzut09 & $71{,}67(8)$   & $0{,}12445(10)$& $1{,}06649(32)$& $790{,}00(1)$ \\
		zrzut10 & $67{,}61(10)$  & $0{,}12986(14)$& $1{,}19635(35)$& $883{,}00(1)$ \\
		zrzut11 & $67{,}91(9)$   & $0{,}12945(13)$& $1{,}32580(38)$& $1099{,}30(2)$ \\
		\hline
		
	\end{tabular}
	\label{tab:srednia_aryl_IRL}
\end{table}



\begin{figure}[H]
	\centering
	\includegraphics[angle=0, width= 14cm]{figures/zrzut001.png}
	\caption{Wykres zrzutu 01, który rozpoczynał  pomiar metodą okresu. Na górnym wykresie wyraźnie widać, że w początkowej fazie wysuwania pręta z rdzenia (obszar, zaznaczony na osi czasu za pomocą dwóch pionowych, przerywanych linii), moc reaktora chwilowo spada, zamiast rosnąć. Oznacza to, że pręt musiał być zanurzony małym fragmentem aż pod rdzeń i w trakcie wyciągania go do góry przez chwilę nie zmniejszało się jego efektywne zanurzenie w rdzeniu, a wręcz zwiększało. }
	\label{zrzut001}
\end{figure}


Ze względu na dużą liczbę wykresów dopasowania w części głównej zamieszczono jedynie reprezentatywne przykłady. Pełny zestaw wykresów dla wszystkich przypadków przedstawiono w repozytorium, który został umieszczony na GitHubie pod adresem: \url{https://github.com/Piegusell/In-ynierska}.

\subsection{Dopasowanie wartości eksperymentalnych z krzywą teoretyczną}\label{dopasowanie}

Krzywą teoretyczną efektywności pręta przyjęto na podstawie zależności
zestawionej w tabeli \ref{tabela}, przedstawionej graficznie na rysunku
\ref{efektywność}. Położenie pręta PK6 było identyczne dla obu detektorów.
Jako parametr stały przyjęto długość czynnego odcinka pręta regulacyjnego:
\[
L = 1~\text{m}.
\]

Dopasowanie charakterystyki teoretycznej do danych eksperymentalnych wykonano
z wykorzystaniem programu napisanego w języku \textit{Python}, stosując procedurę
\texttt{curve\_fit}, opartą na metodzie nieliniowych najmniejszych kwadratów.
Jako parametry dopasowywane przyjęto:
\begin{itemize}
	\item $H$ — efektywną wysokość rdzenia reaktora,
	\item $a$ — współczynnik skali dopasowujący amplitudę modelu do danych
	eksperymentalnych.
	\item \text{$z_{0}$} — początkowe położenia pręta kontrolnego.
\end{itemize}


Pozostałe wielkości występujące w modelu traktowano jako znane i stałe.

Jakość dopasowania oceniono na podstawie analizy reszt
\begin{equation}
	r_i = \rho_i - \hat{\rho}_i,
\end{equation}
gdzie $\rho_i$ oznacza wartości eksperymentalne, natomiast $\hat{\rho}_i$
wartości wyznaczone z modelu teoretycznego dla tych samych położeń pręta.
Zdefiniowano sumę kwadratów reszt (SSE) oraz całkowitą sumę kwadratów (SST):
\begin{equation}
	\mathrm{SSE} = \sum_{i=1}^{N} r_i^2,
	\qquad
	\mathrm{SST} = \sum_{i=1}^{N} \left(\rho_i - \bar{\rho}\right)^2,
\end{equation}
gdzie $\bar{\rho}$ jest średnią arytmetyczną wartości eksperymentalnych.
Na tej podstawie obliczono współczynnik determinacji:
\begin{equation}
	R^2 = 1 - \frac{\mathrm{SSE}}{\mathrm{SST}}.
\end{equation}
Wynik  w postaci porównania charakterystyki eksperymentalnej z krzywą teoretyczną przedstawiono na rys. \ref{fitowanie_obnizone}. Uzyskane rezultaty wskazują, że efektywna wysokość rdzenia reaktora jądrowego MARIA jest prawdopodobnie mniejsza od wartości przyjętej w raporcie~\cite{ILR}. Ponadto, mając na uwadze, że pierwszy etap analizy mógł prowadzić do zaniżenia wartości skumulowanej reaktywności, należy dopuścić możliwość, że rzeczywista wartość parametru $H$ jest jeszcze mniejsza. Dlatego wykonano dopasowanie, w którym dodatkowym parametrem modelu było przesunięcie osi położenia $z_0$.  Zmienna modelowa przyjmuje postać $z_{\mathrm{rel}} = z - z_0$, gdzie $z_0$ jest parametrem dopasowywanym. Dopasowanie powoduje uzyskanie mniejszej, niż z rys. \ref{strumień}, wartości H, bo eksperyment z rys. \ref{strumień} był w wyidealizowanych warunkach, bez zanurzonych prętów w okolicy pomiaru, gdzie podczas pomiaru wykonujemy z zanurzonymi prętami (sam mierzony pręt jest zanurzony i zaburza lokalnie strumień neutronów, zaniżając wartość H).~\cite{ILR}.


\begin{table}[H]	
	
	\centering
	\caption{Funkcja efektywności pręta sterującego użyta w modelu dopasowania}
	\label{funkcja efe}
	\renewcommand{\arraystretch}{1.4}
	\[
	\varepsilon(z_\mathrm{rel}) =
	\begin{cases}
		0,
		& z_\mathrm{rel} \le -\dfrac{H}{2} \\[6pt]
		
		\dfrac{H}{\pi}
		\left[
		\sin\!\left(\dfrac{\pi z_\mathrm{rel}}{H}\right) + 1
		\right],
		& -\dfrac{H}{2} < z_\mathrm{rel} < -\dfrac{H}{2} + L \\[10pt]
		
		\dfrac{H}{\pi}
		\left[
		\sin\!\left(\dfrac{\pi z_\mathrm{rel}}{H}\right)
		-
		\sin\!\left(\dfrac{\pi (z_\mathrm{rel}-L)}{H}\right)
		\right],
		& -\dfrac{H}{2} + L \le z_\mathrm{rel} < \dfrac{H}{2} \\[10pt]
		
		\dfrac{H}{\pi}
		\left[
		1 -
		\sin\!\left(\dfrac{\pi (z_\mathrm{rel}-L)}{H}\right)
		\right],
		& \dfrac{H}{2} \le z_\mathrm{rel} < \dfrac{H}{2} + L \\[10pt]
		
		0,
		& z_\mathrm{rel} \ge \dfrac{H}{2} + L
	\end{cases}
	\]

\end{table}
\begin{table}[H]
	\centering
	\caption{Model matematyczny użyty do dopasowania danych pomiarowych}
	\renewcommand{\arraystretch}{1.3}
	\begin{tabular}{|p{6cm}|c|p{6.5cm}|}
		\hline
		Wielkość & Symbol & Zależność \\ \hline
		
		Położenie względne pręta
		& $z_\mathrm{rel}$
		& $z_\mathrm{rel} = z - z_0$ \\ \hline
		
		Funkcja efektywności pręta
		& $\varepsilon(z_\mathrm{rel})$
		& tabela \ref{funkcja efe}  \\ \hline
		
		Model 
		& $\rho(z)$
		& $\rho(z) = a \, \varepsilon(z - z_0)$ \\ \hline
		
	\end{tabular}
\end{table}






\begin{figure}[H]
	\centering
	\includegraphics[width=14cm]{figures/fitowanie_wynikowe_ob.png}
	\caption{Dopasowanie modelu efektywności pręta kontrolnego z uwzględnieniem przesunięcia początku wsuwania ($z_0$).}
	\label{fitowanie_obnizone}
\end{figure}

\begin{table}[H]
	\centering
	\caption{Wyniki dopasowania modelu efektywności pręta kontrolnego z uwzględnieniem przesunięcia początku wsuwania dla rys.\ref{fitowanie_obnizone}}
	\renewcommand{\arraystretch}{1.2}
	\begin{tabular}{|l|c|c|c|}
		\hline
		Wielkość & Symbol & Wartość & Jednostka \\
		\hline
		Współczynnik skali (dopasowany) 
		& $a$ 
		& $0.00205952 \pm 0.00002986$ 
		& -- \\
		\hline
		Efektywna wysokość rdzenia (dopasowana) 
		& $H$ 
		& $1007.49 \pm 19.22$ 
		& mm \\
		\hline
		Przesunięcie początku wsuwania 
		& $z_0$ 
		& $569.73 \pm 4.06$ 
		& mm \\
		\hline
		Suma kwadratów reszt 
		& $SSE$ 
		& $1.20 \times 10^{-3}$ 
		& \$\textsuperscript{2} \\
		\hline
		Całkowita suma kwadratów 
		& $SST$ 
		& $1.7911$ 
		& \$\textsuperscript{2} \\
		\hline
		Współczynnik determinacji 
		& $R^2$ 
		& $0.99933$ 
		& -- \\
		\hline
	\end{tabular}
	\label{tab:fit_results_z0}
\end{table}