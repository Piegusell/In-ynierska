„Rdzeń reaktora składa się z ciśnieniowych kanałów paliwowych, prętów regulacyjnych
i matrycy złożonej z bloków berylowych. Wokół rdzenia umieszczone są bloki grafitowe spełniające rolę reflektora. Całość umieszczona jest w obudowie zwanej koszem. Kosz posadowiony jest na specjalnej podstawie umieszczonej na dnie basenu reaktora. ” \cite{raport5}

W niniejszej pracy skupiono się głównie na prętach kontrolnych (\textbf{PK}), które stanowią podstawowy element układu regulacji poziomu mocy reaktora. Jak można zauważyć na rysunku \ref{rdzen}, rdzeń reaktora ma kształt stożka, przez co również pręty są wprowadzane pod kątem względem jego osi, co wpływ na regulację \textbf{PK} podczas eksploatacji.

\begin{figure}[H]
	\centering
	\includegraphics[scale = 0.7]{figures/geometria.png}
	\caption{Przekrój poprzeczny bloku reaktora z raportu bezpieczeństwa \cite{raport5}}
	\label{rdzen}
\end{figure}

\newpage
\subsection{Pręty kontrolne}
Pręt kontrolny wykonany jest z substancji pochłaniającej neutrony (węglika boru), zamkniętej w koszulce aluminiowej. Celem konstrukcji pręta jest kontrola reakcji rozszczepienia zachodzącej w rdzeniu reaktora poprzez wydajne pochłanianie neutronów. Zadania stawiane prętom kontrolnym są następujące:
\begin{itemize}
	\item utrzymanie głębokiej podkrytyczności, czyli stanu wyłączenia reaktora,
	\item podnoszenie mocy i praca przy mocy znamionowej,
	\item wyłączanie reaktora poprzez obniżanie mocy,
	\item awaryjne wyłączenie reaktora w przypadku zadziałania systemów zabezpieczeń.
\end{itemize}

Pręt kontrolny może pełnić trzy funkcje: kontrolną, bezpieczeństwa lub automatycznej regulacji.
Kanały prętów bezpieczeństwa (\textbf{PB}) i kompensacyjnych (\textbf{PK}), w tym także pręta automatycznej regulacji (\textbf{PAR}), umieszczone są w blokach berylowych. \cite{raport5}

Materiałem czynnym prętów jest węglik boru (B$_4$C), zwykle z podwyższoną zawartością izotopu $^{10}$B. Pochłanianie neutronów zachodzi głównie na reakcji wychwytu:
\[
^{10}\mathrm{B}+n \rightarrow {}^{7}\mathrm{Li}^{*}+\alpha+2{,}31~\mathrm{MeV}\quad (\approx 93{,}7\%),\ ,
\]
oraz, rzadziej,
\[
^{10}\mathrm{B}+n \rightarrow {}^{7}\mathrm{Li}+\alpha+2{,}79~\mathrm{MeV}\quad (\approx 6{,}3\%).
\]
  Po każdym akcie pochłonięcia neutronu przez $^{10}$B, jądro $^{10}$B jest zastępowane przez $^{7}$Li, a to oznacza, że z czasem $^{10}$B w danym pręcie pochłaniającym jest coraz mniej, czyniąc pręt mniej wydajnym.
  Dlatego tak ważna jest okresowa kontrola prętów poprzez pomiar, wykonany \textbf{metodą zrzutu} (rozdział\,\ref{metoda_zrzutu}) lub \textbf{metodą okresu} (rozdział\,\ref{metoda_okresu}).
