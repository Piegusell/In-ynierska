W tym rozdziale, korzystając z modelu kinetyki punktowej przedstawionego wcześniej (rów.~\eqref{nodt}–\eqref{ciodt} oraz ujęcia macierzowego z macierzą $\mathbf{M}(t)$), wyprowadzimy zależność wiążącą reaktywność $\rho$ z okresem reaktora $T$.

Okres reaktora $T$ definiuje się jako czas, w którym strumień (moc) neutronowa rośnie $e$-krotnie: $n(t)=n_0\,e^{t/T}$. 
W praktyce często podaje się \emph{okres podwojenia} $T_2$, określony warunkiem $n(t)=2n_0$, przy czym
\[
T_2 = T\,\ln 2.
\]

\noindent\textbf{Równanie odwrotnych godzin} opisuje zależność reaktywności reaktora $\rho$ od jego okresu $T$. 
Nazwa pochodzi od „odwrotnej godziny” (ang.\ \emph{inverse hour}, \emph{inhour}) — jednostki reaktywności odpowiadającej sytuacji, w której okres reaktora $T$ wynosi jedną godzinę:
\begin{equation}
	\rho_{in}=\frac{\Lambda + \sum\limits_{i=1}^j \sfrac{\beta_i}{\lambda_i}}{3600}\,\rho
	\label{IN}
\end{equation}

\noindent gdzie:
\begin{itemize}
	\item $\rho$ — reaktywność reaktora,
	\item $\beta_i$ — udział neutronów opóźnionych i fotoneutronów z $i$-tej grupy w całej puli neutronów,
	\item $\Lambda$ — czas życia jednej generacji neutronów,
	\item $\lambda_i$ — stała rozpadu prekursorów neutronów opóźnionych i fotoneutronów z $i$-tej grupy (w~s$^{-1}$).
\end{itemize}

Rozważamy reaktor pracujący do chwili $t=0$ w stanie krytycznym (stacjonarnym), a w chwili $t=0$ wprowadzamy skokowo stałą reaktywność $\rho_0$.
Wobec tego w rozważanym przedziale czasowym przyjmujemy $\rho(t)\equiv \rho_0=\text{const}$, a parametry kinetyczne $(\beta_i,\lambda_i,\Lambda)$ są stałe (charakterystyczne dla danego reaktora).
Odpowiada to szczególnemu przypadkowi macierzowego opisu z poprzedniego rozdziału, w którym:\begin{equation}
	\frac{\mathrm{d}}{\mathrm{d}t} \left[\begin{array}{c}
		n(t)\\
		\uu{C_1(t)}\\
		\uu{ C_2(t)}\\
		\uu{\vdots}\\
		\uu{C_j(t)}
	\end{array}\right]= 
	\underbrace{
		\left[\begin{array}{ccccc}
			\frac{\rho_0-\beta}{\Lambda}&\lambda_1&\lambda_2&\hdots&\lambda_j\\
			\uu{\frac{\beta_1}{\Lambda}}&-\lambda_1&&\text{\huge0}&\\
			\uu{\frac{\beta_2}{\Lambda}}&&-\lambda_2&&\\
			\uu{\vdots}&\text{\huge0}&&\ddots&\\
			\uu{\frac{\beta_j}{\Lambda}}&&&&\lambda_j
		\end{array}\right]
	}_{\ \Resize{0.5cm}{\mathbb{A}}}
	\underbrace{
		\left[\begin{array}{c}
			n(t)\\
			\uu{C_1(t)}\\
			\uu{ C_2(t)}\\
			\uu{\vdots}\\
			\uu{C_j(t)}
		\end{array}\right]
	}_{\ \Resize{0.8cm}{\vec{f}(t)}}
	\label{maciero}
\end{equation}
\noindent  Macierz $\mathbb{A}$ jest kwadratowa i jej wymiar to $(j+1)\times (j+1)$, gdzie $j$ jest liczbą grup neutronów opóźnionych i fotoneutronów. Macierz $\mathbb{A}$ jest także macierzą \textbf{stałą w czasie}.  Oznacza to, że równanie (\ref{maciero}) jest \textbf{jednorodnym} równaniem różniczkowym pierwszego rzędu, typu:


\begin{equation}
	\frac{\mathrm{d}\vec{f}(t)}{\mathrm{d}t}=\mathbb{A}\vec{f}(t)
	\label{jedno}
\end{equation}

\noindent którego rozwiązaniem jest:

\begin{equation}
	\vec{f}(t)=\exp(\mathbb{A}t)\vec{f}(0)
	\label{jednorozw}
\end{equation}

\noindent gdzie:

\begin{equation}
	\vec{f}(0)=\left[\begin{array}{c}
		n(0)\\
		\uu{C_1(0)}\\
		\uu{ C_2(0)}\\
		\uu{\vdots}\\
		\uu{C_j(0)}
	\end{array}\right]
	=\left[\begin{array}{c}
		\uu{n_0}\\
		\uu{C_{1,0}}\\
		\uu{ C_{2,0}}\\
		\uu{\vdots}\\
		\uu{C_{j,0}}
	\end{array}\right]
	\label{fzero}
\end{equation}

\noindent Metoda otrzymania rozwiązania ogólnego równania (\ref{maciero}) przeanalizowana zostanie poniżej w  (podrozdział \ref{inhour}).

