Reaktor MARIA został uruchomiony w grudniu 1974 roku jako część krajowego programu rozwoju energetyki jądrowej. Był to pierwszy reaktor badawczy w Polsce zdolny do pracy w trybie ciągłym, przeznaczony do prowadzenia eksperymentów naukowych, produkcji izotopów promieniotwórczych oraz testowania materiałów jądrowych. 
	
W ciągu kolejnych dekad reaktor przeszedł szereg modernizacji, które zwiększyły jego bezpieczeństwo, wydajność oraz zakres możliwych zastosowań. Obecnie jest jednym z 20 reaktorów badawczych, które produkują izotopy promieniotwórcze w celach medycznych. Aby zachować ciągłość jego pracy i zapewnić nieprzerwaną dostawę preparatów do placówek medycznych, potrzeba regularnych obliczeń i analiz bezpieczeństwa.

\subsection	{Cel i zakres}
Celem niniejszej pracy inżynierskiej jest analiza zachowania reaktora jądrowego w ujęciu kinetyki punktowej, w odpowiedzi na zmianę reaktywności, wywołaną ruchem pręta regulacyjnego.
W pracy zostaną przeprowadzone symulacje numeryczne z wykorzystaniem języka Python, polegające na rozwiązaniu układu równań kinetyki punktowej dla przypadku wprowadzenia dodatniej porcji reaktywności do rdzenia poprzez wysunięcie fragmentu pręta regulacyjnego.
Jak i zostanie wyznaczona charakterystyka efektywności pręta regulacyjnego w funkcji jego zagłębienia w rdzeniu.
Zostanie również wyznaczona wyprowadzone zostanie tzw. „równanie odwrotnych godzin” jako szczególny przypadek rozwiązania równań kinetyki punktowej, przy założeniu wprowadzenia do rdzenia dodatniej porcji reaktywności. Równanie to zostanie wykorzystane w metodzie przeprowadzania pomiaru efektywności pręta regulacyjnego.
Na podstawie tego pomiaru zostanie wyznaczona eksperymentalna charakterystyka efektywnościowa pręta w funkcji jego zagłębienia w rdzeniu, którą następnie porówna się z wcześniej uzyskaną krzywą teoretyczną.
	


