\subsection{Nieobrobione dane z programu PRETY}
Na wykresie przedstawiono surowe dane z programu PRETY. Widoczny jest podwójny charakter przebiegu czasowego: 
w pierwszej części pomiar został wykonany metodą okresu (zob. \ref{metoda_okresu}), 
natomiast w drugiej części zastosowano metodę zrzutu (zob. \ref{metoda_zrzutu}), 
która została następnie poddana analizie \ref{analiza} oraz dopasowaniu modelowemu (\ref{fitowanie_obnizone}) w rozdziale \ref{dopasowanie}.
\begin{figure}[H]
	\centering
	\includegraphics[width=\textwidth]{figures/Dane_nzrobione.png}
	\caption{Nieobrobione przebiegi czasowe sygnałów zarejestrowanych w systemie PRETY. Górny panel przedstawia surowe sygnały z komór rozszczepieniowych IRL 2.1 i IRL 2.2. Dolny panel przedstawia położenie prętów kontrolnych, pręta bezpieczeństwa i pręta automatycznej regulacji. Zielonym kolorem zaznaczono położenia ważonego pręta (PK6), co widać w postaci narastających schodków w prawej części przebiegów. Na pomarańczowo zaznaczono położenia pręta, który kompensował zmiany reaktywności podczas pomiaru (PK2) i widać to, jako pomarańczowe schodki schodzące w dół.}
	\label{fig:prety_surowe}
\end{figure}